
%%%%%%%%%%%%%%%%%%%%%%%%%%%%%%%%%%%%%%%%%%%%%%%%%%%%%%%%%%%%
% \section{Value Co-Creation}
% \begin{comment}
% - What is Value Co creation?
% - Why is it generally important?
% - Why is it important for this thesis? 
%   Identified intentions from clients in video-calls (through their emotions) allow a consultant/chat-bot to adapt/personalize their responses towards that client. Who in return again gives feedback through his/her emotions -> intentions. -> cycle
% \end{comment}

% The co-creation of value is a core concept inside the Service-Dominant Logic (S-D Logic) that every client interaction can be described as a service. In this logic, the service's value is co-created by experiences between the service provider and the service user. \citep{Payne:2008:ValueCo-Creation}
% \newline\newline
% In comparison to S-D Logic where value is embedded in experiences, in the 'traditional' Goods-Dominant Logic (G-D Logic) value is embedded in goods and thus is only created by the produce of goods, but not by the user/client. \citep{Payne:2008:ValueCo-Creation}
% \newline\newline
% As a result, S-D Logic places the client at the same level of companies as the co-creator of value. As both of them engage in dialog through the product design and delivery stage, thus co-creating value through customized products. This co-creation can take place, for example, through the emotional engagement of the customer, as well as the co-design of product or even the transfer of labor to the customer (e.g. IKEA products). \citep{Payne:2008:ValueCo-Creation}
% \newline\newline
% \textbf{How does value co-creation manifest itself in this thesis?}