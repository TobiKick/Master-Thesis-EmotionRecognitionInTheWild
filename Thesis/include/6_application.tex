
\chapter{Application}
The goal for PPI AG with the Emotion Recognition approach proposed in this thesis, is to use it a scientific basis for an application in their business area. Thus, the idea is to assess in this chapter whether such an real-life application for the assistance of consultants during video-calls is viable. In order to provide consultants insightful assistance, it is necessary to go beyond the mere emotions and provide a meaningful metric. It was chosen to use interest as the metric for identifying the intention of a customer.
\newline\newline
In order to assess the viability of such an application in real-time video calls, it is necessary to develop an identification approach based upon current literature, design an experiment and compare the achieved results. A detailed explanation of these points will be provided in the following sections, starting with the identification approach.

\section{Identification approach}
%Is it viable to use Emotion Recognition to analyse consultations during video call in real-time in order to draw conclusions about human intentions (e.g. interest)?
% Unlike in the FaceReader application which bases their Interest prediction on action units, the selected approach is based upon an idea from a 2016 paper \citep{Kamaruddin:2016:MeasuringCustomerSatisfaction} The idea is to determine the level of interest/appreciation only through considering how positive or negative the average value of valence is.
% \newline\newline
% Furthermore, the authors classified their VA-values into four emotion categories and made use of a threshold to determine a neutral emotion category. Inspired by this approach, a threshold will be used to better identify whether a person is interested, neutral, or uninterested.
% \newline\newline
% However interesting from the FaceReader application \citep{Noldus:2020:Facereader} is, that they use an interval o f2 seconds to calculate their value of 'Interest'. This means that the currently calculated value for interest is based on video-frames from the last two seconds.

% This Master thesis is based on the utilized data set and thus is taking over the underlying assumption that facial expressions equate emotions. 
% However, in this application scenario new data is utilized that is not based upon this assumption. As the label, an actual purchase, doesn't care about that.

\section{Experimental setup \& results}
Goal is to compare the test person's perceived subjective interest of a (sales) video-clip with the interest predicted by the before mentioned approach from the web-cam video. Thus, an experiment needs to be set up that allows to record a video from the test person's webcam. Furthermore, a mechanism for obtaining the test person's feedback in terms of interest needs to devised.
\newline\newline
Considerations about different experiment setups are presented in the following table. Hereby, an 'X' indicates the non-fulfillment of an indispensable quality of an approach, while '/' indicates a constraint and ':)' indicates that the quality is fully fulfilled by the approach.
% Please add the following required packages to your document preamble:
% \usepackage[table,xcdraw]{xcolor}
% If you use beamer only pass "xcolor=table" option, i.e. \documentclass[xcolor=table]{beamer}
\begin{table}[H]
\begin{tabular}{ccccc}
\hline
\textbf{} & \begin{tabular}[c]{@{}c@{}}Web-Cam\\ Video\end{tabular} & \begin{tabular}[c]{@{}c@{}}Continuous\\ interest rating\end{tabular} & \begin{tabular}[c]{@{}c@{}}Easily\\ deployable\end{tabular} & \begin{tabular}[c]{@{}c@{}}Results\\ retrievable\end{tabular} \\ \hline
\textbf{\begin{tabular}[c]{@{}c@{}}Questionnaire\\ (e.g. Google Forms)\end{tabular}} & {\color[HTML]{FE0000} \textbf{X}} & {\color[HTML]{FE0000} \textbf{X}} & {\color[HTML]{32CB00} \textbf{:)}} & {\color[HTML]{32CB00} \textbf{:)}} \\
\textbf{\begin{tabular}[c]{@{}c@{}}Packaged Python\\ Application\end{tabular}} & {\color[HTML]{32CB00} \textbf{:)}} & {\color[HTML]{32CB00} \textbf{:)}} & {\color[HTML]{FE0000} \textbf{X}} & {\color[HTML]{FFC702} \textbf{/}} \\
\textbf{\begin{tabular}[c]{@{}c@{}}Web-based Python\\ Application - Server side\\ (hosted by PythonAnywhere)\end{tabular}} & {\color[HTML]{FE0000} \textbf{X}} & {\color[HTML]{32CB00} \textbf{:)}} & {\color[HTML]{32CB00} \textbf{:)}} & {\color[HTML]{32CB00} \textbf{:)}} \\
\textbf{\begin{tabular}[c]{@{}c@{}}Web-based ReactJS\\ Application - Client Side\end{tabular}} & {\color[HTML]{32CB00} \textbf{:)}} & {\color[HTML]{32CB00} \textbf{:)}} & {\color[HTML]{32CB00} \textbf{:)}} & {\color[HTML]{FFC702} \textbf{/}} \\ \hline
\end{tabular}
\caption{Consideration of experimental setups}
\label{tab:appExperiment}
\end{table}




\section{Assessment of viability in real-time video calls}
% especially focus on the time aspect for detecting the bounding box + landmark detection/shape model construction.

% Comparison dlib's shape model vs. Active Appearance Model
% \begin{quote}
%     The dlib 'Face Landmark Detection' algorithm is blazing fast, in fact it takes about 1–3ms (on desktop platform) to detect (align) a set of 68 landmarks on a given face.
% \end{quote} 
% https://medium.com/datadriveninvestor/training-alternative-dlib-shape-predictor-models-using-python-d1d8f8bd9f5c

% @InProceedings{Kazemi_2014_CVPR,
% author = {Kazemi, Vahid and Sullivan, Josephine},
% title = {One Millisecond Face Alignment with an Ensemble of Regression Trees},
% booktitle = {Proceedings of the IEEE Conference on Computer Vision and Pattern Recognition (CVPR)},
% month = {June},
% year = {2014}
% }
