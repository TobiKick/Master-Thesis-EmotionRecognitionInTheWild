
\chapter{Conclusion}
+ Could the research questions be answered? -> How?
+ How are the achievements of ER Applications to be interpreted?

\section{Improvements/Limitation}
AFEW-VA dataset only contains about 30.000 images. The model could be honed with more training data from different datasets and thus even perform better. This could be an improvement step for the actual implementation of such an approach. However, this step was purposefully not taken in this thesis, as it would make an objective comparison with other work very difficult.


\section{Recommendations}
\subsubsection{Next steps after this Master thesis}
=> PPI sales already contacted online banks (ING and DKB) who are very interested into ER to identify whether their costumers are interested/what intention they have.
The idea is to refine the work from this Master thesis with real-life data from the banks online consultations. This video data can be combined with a label, of whether a purchase followed the consultation. This allows to further fine-tune the proposed neural network architecture.

\subsubsection{Further areas of research}
- Making annotations more subject orientated \& culturally independent
e.g. either maybe using an EEG to detect areas of brain activity and mapping these to values of valence and arousal
e.g. actually asking subjects themselves how they felt during a video clip and using this as annotations

\subsubsection{Further fields of application}
=> Call center application for customer satisfaction
=> Enhancement of dynamic content generation through integrating a chatbot who's answers are being guided by the emotions/intentions/interest of a calling person.


\section{Outlook}
+ further fields of research
        -> combination of eye movements and physiological measurements as it was done, for example, in Noldus's product 'Cube' \citep{Noldus:2020:Facereader}
        How can this\textbf{ combination} influence the identification of human behaviors, like interest?