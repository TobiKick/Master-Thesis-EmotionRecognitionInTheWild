
\chapter{Conclusion}
%%%%%%%%%%%%%%%%%%%%%%%%%%%%%%%%%%%%%%%%%%%%%%%%%%%%%%%%%%%%%%%%%%%%%%%%
\section{Hypotheses}
This Master's thesis started off with setting out three hypotheses which are now being scrutinized whether these could be confirmed or disproved.\newline\newline
\textbf{Hypothese \#1: Emotions can be recognized reliably from in-the-wild data used for training and testing. Furthermore, the results can beat the state-of-the-art outcomes demonstrated in current literature.}\newline
Under the assumption, that the AFEW-VA dataset, consiting of talk-show and movie clips, represents authentic in-the-wild data, it was proven that emotions can be recognized reliably. Additionally, the achieved results could even beat results from a state-of-the-art scientific paper.
\newline\newline
\textbf{Hypothese \#2: When examining the implemented modules separately, each module contributes positively towards the performance of the overall architecture.}\newline
Through the preliminary search for optimal hyper-parameter and the rigorous ablation study, this work ensures that each module contributes positively towards the overall performance of emotion recognition in the wild.
\newline\newline
\textbf{Hypothese \#3: In a prototype application, emotion recognition will perform well with real-time video input from a webcam. It will be able to map recognized emotions to human intentions effectively.}\newline
In order to prove this hypothese, a small prototype application was created that could recognize emotions in real-time from the person's face in front of a web camera. Furthermore, a mechanism was devised for mapping emotional valence to a measure of interest. The viability of this mechanism was further validated through a user experiment. The experiment could show a statistically relevant correlation of 0.35 on in-the-wild data gathered from participants watching commercials. As a result of this experiment, it could be proven that recognized emotions can be mapped effectively to human intentions (i.e. interest).



\section{Limitations}
While addressing the above mentioned hypotheses during this work, there were some limitations that had to be considered:

\begin{itemize}
    \item The size of the dataset with about 30.000 frames from 120 subjects can be categorized as rather small, resulting in a higher risk of overfitting the model. Choosing a bigger dataset would of course be an advantage, but is not always available. A solution might also be the combination of multiple datasets to yield better results, as was done by other state-of-the-art papers mentioned in section \ref{sec:StateOfTheArtComparison}.
    \item The selected AFEW-VA database contains in-the-wild video-clips sourced from movies and talk shows which often exhibit emotions strongly and clearly. Ideally, the model would be trained on real-life data obtained from a specific application scenario. However, such data can only be obtained with extreme effort. Therefore, an idea to least improve the variety of data might be to include other databases into the training process, such as the SEMAINE database where participants are recorded while interacting with a human operator.
    \item The Master's thesis is based on the underlying assumption that facial expressions equate emotions. This needs to be kept in mind when analyzing results. A possible way to mitigate this assumption could be the inclusion of data from different sources, e.g. data from an EEG analysis.
    \item The emotions exhibited during video call consultations are generally weak which makes it very hard for the neural network, that was trained on strong facial expressions, to recognize these emotions. This in turn makes it hard to deduct a measure of interest. A solution might be to fine-tune the proposed approach on data from real-life video calls that were labelled with a measure of human intention, such as interest or whether a product was bought in the end.
\end{itemize}

%%%%%%%%%%%%%%%%%%%%%%%%%%%%%%%%%%%%%%%%%%%%%%%%%%%%%%%%%%%%%%%%%%%%
\section{Further areas of research}
During this work it became clear that emotion recognition has a great potential that is accompanied by a lot of open research questions. It would be especially beneficial for this Master's thesis to have a deeper scientifically backed understanding of the following research areas:
\begin{itemize}
    \item \textbf{Combination of multiple databases}:  As outlined during the comparison of the proposed approach with state-of-the-art results, a combination of two or more databases into a training set might substantially increase the model's emotion recognition performance.
    \item \textbf{Combination of multiple input signals}: It is unclear, how different signals are best combined and which signals are best for the identification of human intention, like interest. Such signals include visual, audio, EEG, mouse clicks, eye movements, and physiological measurements.
    \item \textbf{Subject oriented \& culturally independent annotation}: As annotations are often very different across cultures and it is even argued that they are different from subject to subject, research is need on how it is possible to annotate images with these considerations in mind. 
    \item \textbf{Experienced vs. observed emotion}: Annotations are based on observed clues, like facial expressions, while in reality an emotion is depending on the subject's experience. There is more research needed on how to capture the individual experience of subjects during situation instead of relying on professional annotators 'to guess' emotions from visual or audio clues. Some databases already include EEG analysis to observe the experience through brain activity, which is a step in the right direction. However, it should also be considered how far this reflects the experienced emotion by the subject.
\end{itemize}


%%%%%%%%%%%%%%%%%%%%%%%%%%%%%%%%%%%%%%%%%%%%%%%%%%%%%%%%%%%%%%%%%%%%

\section{Application}
\subsection{Next steps for PPI AG}
Based on the prototype application presented in chapter \ref{chap:Application}, the following next steps are recommend for PPI AG in order to adopt emotion reocognition in the real-life use case of video consultations:

\begin{itemize}
    \item A logical first step would be to obtain real-life video-call consultation data for refining the work in this Master's thesis. The idea would be for PPI AG to obtain video consultation data from clients who are already interested into the prospects of this technology. As these companies also know whether for example a purchase followed the consultation, the video data could be combined with this information as a label. This would allow to further fine-tune the proposed neural network architecture.
    \item A second step could be an additional comprehensive experiment about value creation of emotion recognition in real-time video call consultations. As presented in this thesis, it is clear that there is a statistically significant correlation between subjective interest and interest identified through facial emotion recognition. Based on this, a further experiment could clarify how valuable this is to consultants and customers in real-life video calls.
    \item A third step could be the creation of a real-life application that provides tangible value to all participants. Such an application could give customers the ability to rate their experience and give feedback, but also to give hints to the consultant before or during the call on how to improve the customer's interest / satisfaction. The ideal would be that the interest identification brings tangible value to both the consultant and customer, e.g. through additional feedback possibilities, or visualizations.
\end{itemize}

%%%%%%%%%%%%%%%%%%%%%%%%%%%%%%%%%%%%%%%%%%%%%%%%%%%%%%%%%%%%%%%%%%%%
\subsection{Further fields of application}
Next to the explored application of emotion recognition and the identification of interest in real-time video call consultations, further scenarios could include the following:\newline
\begin{itemize}
    \item Utilization of emotion recognition for e-learning courses by giving feedback on the users emotional journeys to e-learning content creators. This could allow them to make content more fun and interesting for students by tailoring their courses based on identified emotions/interest.
    \item Enriching a video-call-bot or chat-bot through emotion recognition plus interest identification. The results obtained from a user's recognized emotions could influence the way the chat-bot responds. This would allow the chat-bot to create a more dynamic and personalized output based on the recognized emotions.
\end{itemize}


%%%%%%%%%%%%%%%%%%%%%%%%%%%%%%%%%%%%%%%%%%%%%%%%%%%%%%%%%%%%%%%%%%%%
\section{Summary}
All in all, this work contributed a detailed review of literature, a methodology to tackle the very complex problem of recognizing emotions from a small in-the-wild dataset, as well as way to identify interest from recognized emotions. Despite many difficulties encountered during this work, it can be be highlighted that this Master's thesis was able to beat the set out emotion recognition baseline results and a state-of-the-art paper. 
\newline\newline
Furthermore, an emotion recognition prototype was successfully constructed that could prove that emotion recognition and interest prediction is possible in real-time without any significant delay. Additionally, the experimental research into the correlation of predicted emotional valence and indicated subjective interest could successfully demonstrate the existence of a statistically relevant correlation between those two variables. These achievements prove that emotion recognition is a very viable technology, ready for practical adoption.
\newline\newline
It can be summarized that this thesis fulfilled the outset objectives by verifying the viability of the proposed approach, contributing new insights to the scientific community, e.g. through the application of Multi-Phase FT, and by applying emotion recognition in a prototype application at PPI AG.