
\chapter{Introduction}

% What makes this topic so important?
Emotion Recognition through the observation of facial expression is a current trend in Artificial Intelligence. Even though big strides are still to be made in this research area, it already finds its way into concrete business applications. As \citet{Chen:2020:EmotionAI} claim, Emotion Recognition is already being deployed in a variety of different scenarios, such as classrooms and courtrooms. They highlight the adoption of Emotion Recognition in AI-assisted hiring software, especially in the US and South Korea, where new applicants have to interview online with an algorithm. These video interviews are then analyzed with Emotion Recognition in order to figure out a candidate's employability score.
\newline\newline
These applications of technological advances in Emotion Recognition show that this technology has a big optimization and innovation potential. This is why the Hamburg-based software and consulting firm PPI AG regards Emotion Recognition as a technology that can play a very important role for its clients in the banking and insurance industry. The cooperation with PPI AG for this Master's thesis brings in another interesting aspect about the viability of technological advances in the field of Emotion Recognition for a possible application In-The-Wild. An initial analysis of the field (in cooperation with PPI AG) determined that video call consultations between customers as the highest-value application scenario for Emotion Recognition, where it could help assist consultants and customers.
\newline\newline
PPI AG, in its commitment to continuously offer innovative solutions to their clients, is looking for new ways to harness technological advances such as Emotion Recognition in order to apply them in real business scenarios. In that vein, PPI AG aims to better serve their customers, create profitable business solutions and to differentiate themselves from competitors. A basis of this builds the scientific examination of the viability of Emotion Recognition In-The-Wild. Given that the field of Emotion Recognition is still relatively new, there are shortcomings in terms of the research available.
\newline\newline
The current gaps in the research within this field are the following:\newline
\begin{itemize}
    \item As supported by \citet{Salah:2018:VideoBasedER} emotions are inherently cultural and personal, making the task of universally interpreting them very difficult. which makes it very difficult to interpret them in a universal way. Even if emotions were universal, there is still the conflict of representing emotions for Machine Learning systems. On the one hand, these representations should carry higher information, but should on the other hand also be easily interpretable by humans.
    \item Another problem with current research is that most research is done in laboratories, under controlled and predictable conditions (e.g., a perfectly-illuminated face of a person).
    \item There is a significant lack of knowledge about fine-tuning Deep Convolution Neural Networks (DCNN). While \citet{Kossaifi:2017:AFEW-VADatabase} found that DCNNs performed worst under all their examined approaches, \citet{Handrich:2020:SimultaneousPredVA}, on the other hand, are showcasing results in favor of FT-DCNNs.
    \item It is unclear which modules and frameworks work specifically for recognizing emotions with In-The-Wild data.
    \item The lack of focus on business opportunities with Emotion Recognition can also be seen as a research gap, as more research is needed on how to utilize emotions for more commercial purposes.
\end{itemize}

While it is possible to recognize emotions from various input signals, such as facial expressions, speech patterns, and keystrokes, this prototype will be focusing primarily on the recognition of emotions from facial expressions. This simplifies the study, and allows for an easier understanding of the components and modules involved in Emotion Recognition. Furthermore, it will allow for a better comparison of the results achieved. 
\newline\newline
% What is the general topic? 
The main objective of this Master's thesis is to implement a prototype that is able to visually recognize human emotions in-the-wild. The goal of this prototype is to act as a demonstration tool for illustrating the commercial benefits of utilizing an Emotion Recognition software, especially during real-time video calls. 
\newline\newline
% What is my research goal/hypothesis? 
The research will be guided by the following hypotheses: 
\begin{itemize}
    \item Emotions can be recognized reliably from In-The-Wild data used for training and testing. Furthermore, the results could even possibly beat the state-of-the-art outcomes demonstrated in current literature.
    % Question 1: How can the proposed approach be compared to state-of-the-art literature and how well does it perform on in-the-wild data?
    % 1) If: Usage of state-of-the-art 3rd party tools -- Then: Recognize emotions as well as the state-of-the-art literature on in-the-wild data
    \item When examining the implemented modules separately, each module contributes positively towards the performance of the overall architecture.
    % Question 2: How do the implemented modules contribute towards the success of the Machine Learning model?
    % 2) If: Modules are considered separately -- Then: single modules contribute positively towards the performance of the overall architecture.
    \item In a prototype application, Emotion Recognition will perform well with real-time video input from a webcam. It will be able to map recognized emotions to human intentions effectively.

    % Question 3: Can the achieved results be applied for identifying human intentions in a prototype application?
    % 3) If: recognizing emotions from a webcam stream -- Then: human intentions identification
\end{itemize}

% How do I prove the above hypothesis? What do I use as a basis/dataset?
% How would my research supplement the flaws/gap in existing research?
Proving these hypotheses will be further discussed in the \textbf{Methodolgy} chapter, where the researcher will be presenting the proposed approach in concrete and palpable steps -- from the image input to its output. At the core of the approach is the Neural Network, which will be explained in detail in the \textbf{Implementation} chapter. The Neural Network will be trained on the In-The-Wild AFEW-VA dataset containing In-The-Wild video clips, mainly from talk shows and movies.
\newline\newline
In the \textbf{Results and Analysis} chapter, achieved outcomes will be discussed and compared to current state-of-the-art results. This will supplement the gaps in existing literature by providing further insights into the process of fine-tuning DCNNs when handling In-The-Wild data. As an extension of this, in the \textbf{Ablation Study} section, a profound analysis is done for the most important modules, in order to illuminate the contributions for each module, separately. The insights obtained through this analysis will contribute towards expanding the knowledge about the positive impact of single modules/frameworks on the performance of Emotion Recognition with In-The-Wild data.
\newline\newline
In the \textbf{Application} chapter, the achieved results will be implemented in an application prototype that can demonstrate real-time viability of such an approach. This will create scientific evidence that will support the maturity of Emotion Recognition for specific business application scenarios.\newline
Moreover, a user experiment will be conducted that compares the predicted interest based on emotions recognized from facial expressions with the stated level of interest by participants. In this way, conclusions can be drawn about a suspected correlation between Emotion Recognition and human interest. This will supplement research gaps on the utilization of emotion information for the identification of human interest. 
\newline\newline
% What will my contribution to the scientific community be?
This study's overall contribution to the scientific community can be summarized by the following three points:
\begin{itemize}
    \item Performance of Multi-Phase Fine-Tuning vs. Single-Phase Fine-Tuning in the field of Emotion Recognition In-The-Wild
    \item Comparison of different ways of dealing with facial landmarks in the field of Emotion Recognition (AAM vs. ASM vs. Landmarks)
    \item Practical insights gained through an experiment on identifying interest by observing people's facial expressions
\end{itemize}