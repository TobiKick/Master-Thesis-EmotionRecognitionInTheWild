\documentclass[11pt,a4paper]{scrbook}
\usepackage{geometry}	
\usepackage[utf8]{inputenc}
\usepackage[T1]{fontenc}
\usepackage[pdftex]{graphicx}
\usepackage[ngerman, english]{babel} % the language specified at the end of the list will be used !!
\usepackage{colortbl}	
\usepackage{soul}
% For APA referencing style
%\usepackage[natbibapa]{apacite}
%\bibliographystyle{apacite}
\usepackage[numbers]{natbib}
\bibliographystyle{plainnat}

\usepackage{textcomp}
\usepackage{booktabs} % for latext table
\usepackage{subfig}

\renewcommand{\familydefault}{\sfdefault}
\usepackage{xcolor}
\definecolor{uhhred}{cmyk}{0,100,100,0}
\usepackage{float}
\usepackage[ddmmyyyy]{datetime}
\usepackage{comment}
\usepackage[acronym,nomain]{glossaries}
\usepackage{enumitem}

\specialcomment{tobi}{\begingroup\sffamily\color{blue}\commentstyle=\color{codegreen},}{\endgroup}
\specialcomment{shan}{\begingroup\sffamily\color{red}}{\endgroup}

% for the legend of the conditions
\usepackage{array,tabularx}
\newenvironment{conditions*}
  {\par\vspace{\abovedisplayskip}\noindent
   \tabularx{\columnwidth}{>{$}l<{$} @{\ : } >{\raggedright\arraybackslash}X}}
  {\endtabularx\par\vspace{\belowdisplayskip}}
  

% Generate the glossary
\makeglossaries
 
%%%%%%%%%%%%%%%%%%%%%%%%%%%%%%%%%%%%%%%%%%%%%%%%%

\newacronym{ML}{ML}{Machine Learning}
\newacronym{EEG}{EEG}{Electroencephalography}
\newacronym{ADR}{ADR}{Action Design Research}
\newacronym{MTCNN}{MTCNN}{Multi-Task Cascaded Convolutional Neural Network}
\newacronym{AAM}{AAM}{Active Appearance Model}

%%%%%%%%%%%%%%%%%%%%%%%%%%%%%%%%%%%%%%%%%%%%%%%%%%

\begin{document}

\frontmatter  % The \frontmatter declaration makes the pages numbered in lowercase roman
\newgeometry{centering,left=2cm,right=2cm,top=2cm,bottom=2cm}
\begin{titlepage}
\includegraphics[scale=0.3]{UHH-Logo_2010_Farbe_CMYK.pdf}
\vspace*{2.0cm}
\Large
\begin{center}
{\color{uhhred}\textbf{\so{MASTER THESIS}}}
\vspace*{1.0cm}\\
{\textbf{at the University of Hamburg in the Department of Computer Science in the master's programme IT Management and Consulting}}
\vspace*{2.0cm}\\
{\LARGE \textbf{Title:}}
\vspace*{0.4cm}\\
{\LARGE Visual Emotion Recognition in The Wild}
%{\LARGE Visual Emotion Recognition in The Wild: Prediction of Valence/Arousal and Assessment of Application scenarios}
%{\LARGE Identifying customer intentions in real-time video consultations through emotion recognition: Analysis of viability and value co-creation by implementation of a prototype}
\vspace*{1.5cm}\\
presented by
\vspace*{0.4cm}\\
Tobias Kick
\end{center}
\vspace*{2.0cm}

\noindent
MIN-Faculty \vspace*{0.25cm} \\
Department of Computer Science\vspace*{0.25cm} \\
Program: IT Management and Consulting\vspace*{0.25cm} \\
Research group: Computer Vision\vspace*{0.25cm} \\
Matriculation number: 7213712 \vspace*{0.5cm} \\
%Submission date: 15/12/2020 \vspace*{0.5cm} \\  
Supervisor: Noha Sarhan \vspace*{0.25cm} \\
First assessor: Prof. Dr. Simone Frintrop \vspace*{0.25cm} \\
Second assessor: Dr. Mikko Lauri

\end{titlepage}

\restoregeometry

%%%%%%%%%%%%%%%%%%%%%%%%%%%%%%%%%%%%%%%%%%%%%%%%%%%%%%%%%%%%


\chapter{Abstract}
The goal of this Master's thesis is to explore the possibilities of the current state-of-the-art in emotion recognition in the wild. A prototype was implemented that aims to possibly break records set in this field. In addition, findings were applied to the business use-case of online video-call consultations.
\newline\newline
For this, the Acted Facial Expressions in The Wild - Valence Arousal (AFEW-VA) dataset \citep{Kossaifi:2017:AFEW-VADatabase}, consisting of talk-show and movie clips, was utilized in this work. Under the assumption that this kind of data represents in-the-wild data, this work confirmed that emotions can be recognized reliably using a Fine-Tuned Deep Convolution Neural Network (FT-DCNN). Furthermore, after optimizing the model and its hyper-parameters, the obtained results could indeed beat the results of one out of two state-of-the-art scientific papers by decreasing the loss in terms of RMSE by 39 \%.
\newline\newline
The thesis will shed a light on the implemented modules through a detailed description and a diligent examination. This examination was done in the form of an ablation study which ensured that all major components contributed positively towards the overall performance of the architecture. The outcome of this scientific examination resulted in a prototype application that can recognize emotions from the webcam stream in real-time.
\newline\newline
An important practical aspect of this thesis is the question, how the results can be applied in the business use case of video consultations. Therefore, a mapping mechanism was devised from emotional valence to interest, and an user experiment was set up in order to validate this approach. In the experiment, participants watched different commercials, during which they were recorded through their webcam. However, facial expressions were way weaker than expected. As a result, the outcome of the experiment disproved the original hypotheses that it is possible to map recognized emotion to human intentions effectively (= with a moderate correlation), at least from video data with weak facial expressions.

% All in all, there are still many open (research) questions to be answered that were revealed by this thesis. One concrete next step could be the utilization of real-life data from video consultations for honing the current model in a way that it better tackles the specific challenge at hand.


%%%%%%%%%%%%%%%%%%%%%%%%%%%%%%%%%%%%%%%%%%%%%%%%%%%%%%%%%%%%

\tableofcontents

\printglossary

%%%%%%%%%%%%%%%%%%%%%%%%%%%%%%%%%%%%%%%%%%%%%%%%%%%%%%%%%%%%
%%%%%%%%%%%%%%%%%%%%%%%%%%%%%%%%%%%%%%%%%%%%%%%%%%%%%%%%%%%%
\mainmatter

%%%%%%%%%%%%%%%%%%%%%%%%%%%%%%%%%%%%%%%%%%%%%%%%%%%%%

\chapter{Introduction}

% What makes this topic so important?
Emotion recognition through the observation of facial expression is a current trend \citep{Chen:2020:EmotionAI}  \citep{Handrich:2020:SimultaneousPredVA} in Artificial Intelligence. Even though big strides are still to be made in this research area, it already finds its way into concrete business applications. \citet{Chen:2020:EmotionAI} claim that emotion recognition is already being deployed in a variety of different scenarios, such as classrooms and courtrooms. They highlight the adoption of emotion recognition in AI-assisted hiring software, especially in the US and South Korea, where new applicants have to interview online with an algorithm. Video interviews are then analyzed with emotion recognition in order to figure out a candidate's employability score.
\newline\newline
These applications of technological advances in emotion recognition show that this technology has a big optimization and innovation potential. This is why the Hamburg-based software and consulting firm PPI AG regards emotion recognition as a technology that can play a very important role for its clients in the banking and insurance industry. The cooperation with PPI AG for this Master's thesis brings in another interesting aspect about the viability of technological advances in the field of emotion recognition for a possible application in the wild. An initial analysis of the field (in cooperation with PPI AG) determined that video call consultations between customers as the highest-value application scenario for emotion recognition, where it could help assist consultants and customers.
\newline\newline
PPI AG, in its commitment to continuously offer innovative solutions to their clients, is looking for new ways to harness technological advances such as emotion recognition in order to apply them in real business scenarios. In that vein, PPI AG aims to better serve their customers, create profitable business solutions and to differentiate themselves from competitors. A basis of this builds the scientific examination of the viability of emotion recognition in the wild. Given that the field of emotion recognition is still relatively new, there are shortcomings in terms of the research available.
\newline\newline
The current gaps in the research within this field are the following:\newline
\begin{itemize}
    \item As supported by \citet{Salah:2018:VideoBasedER} emotions are inherently cultural and personal, making the task of universally interpreting them very difficult. Even if emotions were universal, there is still the conflict of representing emotions for machine learning systems. On the one hand, these representations should carry higher information, but should on the other hand also be easily interpretable by humans.
    \item Another problem with current research is that most research is done in laboratories, under controlled and predictable conditions (e.g., a perfectly-illuminated face of a person).
    \item There is a significant lack of knowledge about fine-tuning Deep Convolution Neural Networks (DCNN). While \citet{Kossaifi:2017:AFEW-VADatabase} found that DCNNs performed worst under all their examined approaches, \citet{Handrich:2020:SimultaneousPredVA}, on the other hand, are showcasing results in favor of FT-DCNNs.
    \item It is unclear which modules and frameworks work specifically for recognizing emotions with in-the-wild data.
    \item The lack of focus on business opportunities with emotion recognition can also be seen as a research gap, as more research is needed on how to utilize emotions for more commercial purposes.
\end{itemize}

While it is possible to recognize emotions from various input signals \citep{Akcay:2020:SpeechEmotionRecognition(SER)}, such as facial expressions, speech patterns, and keystrokes, this prototype will be focusing primarily on the recognition of emotions from facial expressions. This simplifies the study, and allows for an easier understanding of the components and modules involved in emotion recognition. Furthermore, it will allow for a better comparison of the results achieved. 
\newline\newline
The main objective of this Master's thesis is to implement a prototype that is able to visually recognize human emotions in the wild. In order to proof that, the outcomes of the implemented prototype application will be compared objectively to state-of-the-art results in current literature.
\newline\newline
Additionally, implemented design choices will be examined separately on whether they contribute positively towards the overall performance of the trained neural network (i.e. ablation study). Among others, different fine-tuning approaches will be compared; most notably Multi-Phased Fine-Tuning (Multi-Phased FT). Instead of training the whole pre-trained network from the beginning (here denoted as Single-Phased Fine-Tuning), the Multi-Phased FT starts off with training only a few layers and adds to them step-wise more layers.
\newline\newline
Another area of special scientific interest are facial landmarks and its ensuing performance changes for emotion recognition. Detected landmarks will be applied in various ways, ranging from a fully-visible overlay to a separate mask that is fed as a fourth image channel into the neural network.
\newline\newline
Moreover, the goal of this prototype is to act as a demonstration tool to illustrate commercial benefits of utilizing emotion recognition, especially during real-time video calls. For this, the prototype will prove its real-time viability by processing video input from a webcam. For the assessment of commercial benefit, a mechanism will be created to map recognized emotions to a level of interest.
\newline\newline
Hereinafter a short introduction to the chapters in this thesis is given:\newline
The \textbf{Methodolgy} chapter presents the proposed approach in concrete and palpable steps -- from the image input to its output. At the core of the approach is the neural network, which will be explained in detail in the \textbf{Implementation} chapter. The neural network will be trained on the Acted Facial Expressions in the wild - Valence Arousal (AFEW-VA) dataset containing in-the-wild video clips, mainly from talk shows and movies.
\newline\newline
In the \textbf{Results and Analysis} chapter, achieved outcomes will be discussed and compared to current state-of-the-art results. This will supplement the gaps in existing literature by providing further insights into the process of fine-tuning DCNNs when handling in-the-wild data. As an extension of this, in the \textbf{Ablation Study} section, a profound analysis is done for the most important modules, in order to illuminate the contributions for each module, separately. The insights obtained through this analysis will contribute towards expanding the knowledge about the positive impact of single modules/frameworks on the performance of emotion recognition with in-the-wild data.
\newline\newline
In the \textbf{Application} chapter, the achieved results will be implemented in an application prototype that can demonstrate real-time viability of such an approach. This will create scientific evidence that will support the maturity of emotion recognition for specific business application scenarios.\newline
Moreover, a user experiment will be conducted that compares the predicted interest based on emotions recognized from facial expressions with the stated level of interest by participants. In this way, conclusions can be drawn about a suspected correlation between emotion recognition and human interest. This will supplement research gaps on the utilization of emotion information for the identification of human interest. 
\newline\newline
% What will my contribution to the scientific community be?
This study's overall contribution to the scientific community can be summarized by the following three points:
\begin{itemize}
    \item Performance of Multi-Phase Fine-Tuning vs. Single-Phase Fine-Tuning in the field of emotion recognition in the wild
    \item Comparison of different ways of dealing with facial landmarks in the field of emotion recognition
    \item Practical insights gained through an experiment on identifying interest by observing people's facial expressions
\end{itemize}

\chapter{Related Work}

\section{Emotion representation}
A still widely discussed research question in Emotion Recognition is the question about how to best represent emotions or how to model affect. \citet{Gunes:2011:EmotionRepresentationContinuous} state that current approaches stem from research in psychology, which distinguishes between three major approaches: \textbf{categorical, dimensional and appraisal-based.}
% According to research in psychology, three major approaches to affect modelling can be distinguished [1]: categorical, dimensional and appraisal-based approach. The categorical approach claims that there exist a small number of emotions that are basic, hard-wired in our brain and recognised universally (e.g., [2]). This theory on universality and interpretation of affective nonverbal expressions in terms of basic emotion categories has been the most commonly adopted approach in research on automatic measurement of human affect.
% \citep{Gunes:2011:EmotionRepresentationContinuous}
\newline\newline
The \textbf{categorical approach} classifies emotions into a small number of discrete categories which are universally recognised. Usually, these are comprised of six or seven categories, which include 'happy', 'sad', 'fear', 'anger', 'disgust', 'surprise' and sometimes also 'neutral' \citep{Hupont:2010:FacialEmotionsIn2DAffectiveSpace}. Even though there are many other ways to represent emotions, \citet{Salah:2018:VideoBasedER} argue, that this approach is still very relevant today, as its more natural for humans to interpret emotions in that way. This is supported by \citet{Gunes:2011:EmotionRepresentationContinuous} who say that the theory on universality, as well as the facility of interpretation has made this approach the most commonly adopted in research on automatic measurement of human affect.
\newline\newline
However, \citet{Gunes:2011:EmotionRepresentationContinuous} argues that this categorical approach is not able to capture the complexity of the affective state exhibited by people in a rather complex and subtle ways, like embarrassment.
% However, a number of researchers have shown that in everyday interactions people exhibit non-basic, subtle and rather complex affective states like thinking, embarrassment or depression. Such subtle and complex affective states can be expressed via dozens of anatomically possible facial and bodily expressions, audio or physiological signals. Therefore, a single label (or any small number of discrete classes) may not reflect the complexity of the affective state conveyed by such rich sources of information
% \citep{Gunes:2011:EmotionRepresentationContinuous}
\newline\newline
%The most widely used dimensional model is a circular configuration called Circumplex of Affect introduced by Russell [3]. This model is based on the hypothesis that each basic emotion represents a bipolar entity being a part of the same emotional continuum. The proposed polars are arousal (relaxed vs. aroused) and valence (pleasant vs. unpleasant)
% \citep{Gunes:2011:EmotionRepresentationContinuous}
A widely used \textbf{dimensional approach} called 'Continuous Affective Space' was proposed by \citet{Hupont:2010:FacialEmotionsIn2DAffectiveSpace} and consists of representing emotions as a point in a two-dimensional plane instead of discrete categories. Thus, they represent an emotion as a combination of two values, valence and arousal. While valence indicates how positive or negative an emotion is, arousal indicates how calming or exciting an emotion is. \citet{Hupont:2010:FacialEmotionsIn2DAffectiveSpace} claim that this approach allows them to to capture more information by considering intermediate emotional states.
% Hupont, Cerezo, and Baldassarri(2010): Continous Affective Space, mapping emotions into a 2D space allowing them to consider intermediate emotional states (intermediate emotional state = in-between emotions, gradients of emotions e.g., anxiety to fear, horror to disgust [presumably]). This study allowed for the representation of emotions as a point in a plane, instead of mere items in a categorical list.
% Valence (whether emotion is + or -), arousal (strength of the emotion of expression)
% \citep{Hupont:2010:FacialEmotionsIn2DAffectiveSpace}
\newline\newline
However, \citet{Salah:2018:VideoBasedER} object that a complex emotion can be reduced to a point in a valence-arousal region. She uses 'love' to exemplify that an emotion can be attributed with a positive or negative value depending on the context. As we would usually identify 'love' with a positive valence, a concerned expression of a mother looking at her sick child could be mapped to a negative valence. Therefore, \citet{Salah:2018:VideoBasedER} argue that there is still a need for mapping such points to a semantic space, so that humans can properly interpret the emotion.
% However, categorical and discrete approaches that go beyond the six (or seven, if we include “contempt”) basic expressions are still very relevant, as their interpretation is more natural for humans. Also, it is difficult to reduce a complex emotion to a point or region in the valence-arousal space. For instance, we can claim that “love” is a positive emotion, and has a positive valence, but this is not always the case. Consider the loving, concerned expression of a mother, looking at her sick child, and this becomes obvious. Subsequently, a given image or video can be annotated in the continuous space, but there is still a need for mapping such points to a semantic space, where it can be properly interpreted 
% \citep{Salah:2018:VideoBasedER}
\newline\newline
A further \textbf{dimensional approach} is built upon the previously described approaches and argue that even the two dimensional affective space with its valence and arousal is insufficient to represent the variety of emotions accurately. There exist different approaches, like the Pleasure, Arousal and Dominance (PAD) approach \citep{Gunes:2011:EmotionRepresentationContinuous} and the Valence, Arousal and Dominance (VAD) approach \citep{Verma:2017:3D-VAD}
% Another well-accepted and commonly used dimensional description is the 3D emotional space of pleasure – displeasure, arousal – nonarousal and dominance – submissiveness [4], at times referred to as the PAD emotion space [6] or as emotional primitives [7]
% \citep{Gunes:2011:EmotionRepresentationContinuous}
% The valence-arousal model is insufficient to represent emotions accurately. Thus, they introduce a 3D model, called Valence-Arousal-Dominance.
% \citep{Verma:2017:3D-VAD}
\newline\newline
\citet{Gunes:2011:EmotionRepresentationContinuous} points out that a major challenge for utilizing affective data is it's annotation process, as there exists no general annotation scheme agreed upon by all researchers. Therefore, it cannot be excluded, that the annotations possess a personal bias according to the human annotator's personal context and cultural background. Developing such an annotation scheme would need to be unambiguous and facilitate inter-observer agreement.
% A major challenge in affective data annotation is the fact that there is no coding scheme that is agreed upon and used by all researchers in the field that can accommodate all possible communicative cues and modalities. Development of an easy to use, unambiguous and intuitive annotation scheme that is able to incorporate inter-observer agreement levels will indeed ease the heavy burden of the annotation task. Obtaining high inter-observer agreement is another challenge in affect data annotation, especially when (continuous) dimensional approach is adopted.
% \citep{Gunes:2011:EmotionRepresentationContinuous}
\newline\newline
The \textbf{appraisal-based approach} is still considered by \citep{Gunes:2011:EmotionRepresentationContinuous} as an open research question for automatic measurement of affect. The approach consists in generating emotions through continuous subjective evaluation of the subject's internal state and the state of the outside world.
% In the appraisal-based approach emotions are generated through continuous, recursive subjective evaluation of both our own internal state and the state of the outside world (relevant concerns/needs) [1], [5], [8], [10]. Despite pioneering efforts of Scherer and colleagues (e.g., [11]), how to use the appraisal-based approach for automatic measurement of affect is an open research question as this approach requires complex, multicomponential and sophisticated measurements of change.
% \citep{Gunes:2011:EmotionRepresentationContinuous}


%%%%%%%%%%%%%%%%%%%%%%%%%%%%%%%%%%%%%%%%%%%%%%%%%%%%%%%%%%%%%%%%%%

\section{Emotion Recognition}
\begin{quote}
    Emotion recognition refers in psychology to the attribution of emotional states based on the observation of visual and auditory nonverbal cues. \citep{Baenziger:2014:MeasuringERAbility}
\end{quote}
In Computer Science, however, it often seems as if researchers directly equate facial expressions with peoples emotions. \citet{Barrett:2019:EmotionalFromFacialMovements} claim that based upon a review of psychological research, the belief of reliably corresponding facial expressions with emotions is unfounded. This is especially true for the categorical approach of emotion representation, as it favors human interpretability over precision. Therefore, it just assumes that only six or seven basic emotions exist which can be classified by the observation of visual and auditory nonverbal cues.
\newline\newline
Nevertheless, it has to be pointed out, that even this Master thesis, by only looking at facial expressions and by utilizing the AFEW-VA dataset with its annotations, has a inherent assumption that facial expressions correspond to a person's emotions.\newline
This raises the following question: How much of a person's facial expression actually corresponds to the real emotion?
\newline\newline
\citet{Barrett:2019:EmotionalFromFacialMovements} admit that scientific evidence confirms that people do sometimes smile when happy, frown when sad or scowl when angry. This is true for more cases than would be assumed by chance. The authors clarify that people scowl, on average, less than 30 percent of the time when they angry. Conversely, 70 percent of the time people are not scowling when they are angry. Hence, scowls are only one of many expression of anger. Assuming that anger is only detected when a person scowls, the detection rate is with, assuming 30 percent, really low. It is even worse when considering that not every scowl implies the emotion of anger. This is backed up by \citet{Barrett:2019:EmotionalFromFacialMovements} stating that similar facial movements can express multiple instances of different emotion categories.
% The available scientific evidence suggests that people do sometimes smile when happy, frown when sad, scowl when angry, and so on, as proposed by the common view, more than what would be expected by chance.
% People, on average, the data show, scowl less than 30 percent of the time when they’re angry,” says Barrett. “So scowls are not the expression of anger; they’re an expression of anger — one among many. That means that more than 70 percent of the time, people do not scowl when they’re angry. And on top of that, they scowl often when they’re not angry.”
\newline\newline
Furthermore, facial expressions are interpreted differently by various cultures. \citet{Salah:2018:VideoBasedER} state that in order to establish a ground truth for a facial expression, the cultural background for the subject and the annotator is needed. Thus, if there are two annotators from two different cultural backgrounds, the ratings may differ substantially from each other. The authors \citep{Salah:2018:VideoBasedER} cite an experiment with American and Japanese subjects that were asked to annotate facial expression. Even with a closed set of discrete labels the agreement rate was as low as 54.5\% for "fear" and 64.2\% for "anger" annotations. These results show that there is a strong bias of the annotator in every dataset. However, whether this is good or bad depends according to \citet{Salah:2018:VideoBasedER} on whether we want the application's algorithm to learn that bias.
% For emotion estimation, we have ample empirical evidence that different cultures interpret facial expressions differently [39]. According to these findings, establishing ground truth for a facial expression database requires the annotation of both the subject and the annotator’s cultural background. If the same database is annotated by, say, Japanese and American subjects, we may get different ratings, unless a very clear, discrete categorization is used. The typical scenario of letting the annotators choose from a closed set of annotation labels may mask the differences in perception, and even when a closed set is used, we see empirical evidence for these differences. In an early study, Matsumoto indeed experimented with American and Japanese subjects, and obtained as low as 54.5% agreement for “fear” annotations and 64.2% agreement for “anger” annotations in Japanese subjects, even with a closed set of discrete labels for annotation [57]. In any case, we may or may not want the algorithm to learn the biases of the annotators, depending on the application. If an algorithm is pre-screening applicants for a job interview selection decision (a highly undesired situation, but may be conceivable for job posts with tens of thousands of applicants), we do not want any biases there
%\citep{Salah:2018:VideoBasedER}
\newline\newline
\citet{Barrett:2019:EmotionalFromFacialMovements} do not only see differences in respect to facial expression interpretation across cultures, but also across different situations and even across people within a single situation. This can be interpreted in a way, that companies or institutions who use AI to recognize people's emotions and base their decisions on these outcomes, end up misleading their consumers. Therefore, \citet{Barrett:2019:EmotionalFromFacialMovements} are of the opinion that we need to completely rethink our relationship to emotions, as they are varied, complex and situational. They compare the needed change in thinking to Charles Darwin's work on the nature of species, as he recognized that each species is a category of highly variable individuals. The same is regarded as true by \citet{Barrett:2019:EmotionalFromFacialMovements} for emotional categories.
% Yet how people communicate anger, disgust, fear, happiness, sadness, and surprise varies substantially across cultures, situations, and even across people within a single situation. 
% \citep{Barrett:2019:EmotionalFromFacialMovements}
% This, in turn, means companies that use AI to evaluate people’s emotions in this way are misleading consumers.
% \citep{Barrett:2019:EmotionalFromFacialMovements}
% Barrett says that perhaps the most important takeaway from the review is that we need to think about emotions in a more complex fashion. The expressions of emotions are varied, complex, and situational. She compares the needed change in thinking to Charles Darwin’s work on the nature of species and how his research overturned a simplistic view of the animal kingdom. “Darwin recognized that the biological category of a species does not have an essence, it’s a category of highly variable individuals,” says Barrett. “Exactly the same thing is true of emotional categories.”
% \citep{Barrett:2019:EmotionalFromFacialMovements}
\newline\newline\newline
\textbf{Technical state-of-the-art}\newline
Despite all the aforementioned considerations about recognizing emotions from facial expressions, researchers are nevertheless working towards bridging the gap between facial expressions and emotions by finding technical solutions. These solutions are rarely driven by a rethinking as proposed by \citet{Barrett:2019:EmotionalFromFacialMovements}, rather by a strive to improve algorithms and their recognition results. One of these solutions that not only improves current algorithms, but also might help to better understand and interpret facial expressions is called Facial Action Coding System (FACS) \citep{Ekman:2002:FACS}. This system describes visually discernible facial movements and is comprised of a set of codes, also called Action Units (AUs). that describe the presence and intensity of facial muscle movements. Thus, each Action Unit (AU) is described by the presence and intensity of facial muscle movements. As FACS is purely descriptive, it is blind about whether these AUs express any emotions. As a result of the descriptive nature of this system, it might actually help researchers to better interpret facial expressions.
% Facial Action Coding System, or FACS (Ekman, Friesen, & Hager, 2002), is a systematic approach to describe what a face looks like when facial muscle movements have occurred. FACS codes describe the presence and intensity of facial movements. FACS is purely descriptive and is therefore agnostic about whether those movements might express emotions or any other mental event.11 Human coders train for many weeks to reliably identify specific movements called action units (AUs). Each AU is hypothesized to correspond to the contraction of a distinct facial muscle or a distinct grouping of muscles that is visible as a specific facial movement.
% \citep{Barrett:2019:EmotionalFromFacialMovements}
\newline\newline
As Emotion Recognition is always based on the observation of subjects in order to attribute them an emotional state. In an optimal scenario researchers would want to observe subject's emotions directly. As this is currently not doable the closest researchers can get, is the performance of an EEG analysis to measure brain activity. \citet{Xing:2019:EEGAudioVisual} do exactly that by inducing video stimulation in participants and fused those signals with video and audio data. They state that this approach was able to achieve their best results.
\newline
Even though this approach is very promising, the authors complain that there is currently no original EEG-video emotion dataset available where EEG signals are obtained from participants that were induced by the video stimulation. However, such a database inherently has no 'In-The-Wild' conditions, as an EEG would need to be specifically prepared. Moreover, it would be very difficult to apply such models in real-life applications, as an EEG analysis is not readily available for every user/customer.
\newline\newline
Thus, current researchers focus their efforts on boosting the expressiveness of their Emotion Recognition models by basing their predictions on more information. Thus, it is very common that researchers try to find the best combination of signal fusion, like for example, audio with video and mouse movements. These signal combinations are often supported with extracted features that help to efficiently characterize the emotional content of the chosen signals.
\newline\newline
Using Action Units to describe facial muscle movements is such a way of extracting features from video signals. Prior to feature extraction in Emotion Recognition, it is essential to perform face recognition in order to set a bounding box for the person's face. Sometimes the face is recognized by the chosen model architecture at training time, however, when setting a bounding box or detecting landmarks the facial recognition needs to be performed separately. Nowadays, there are publicly available pre-trained algorithms, like the MTCNN algorithm \citep{Zhang:2016:MTCCN} that perform the task of face recognition, bounding boxing and also landmark detection in one go. As stated by \citet{Zhang:2016:MTCCN}, all these three tasks are somewhat correlate, which is why it makes sence to combine them in one three-layered architecture. The achieved results were better than multiple state-of-the-art architectures from the year 2016.
% approach: tasks of face detection, bounding boxing and landmark detection are closely related and somehow dependent on each other. Thus they made use of a three layered CNN architecture where they start to detect faces in the first CNN, go over to set the bounding box and then detect the facial landmarks. All this combined in one three-layered architecture, namely MTCNN - Multi-Task Convolutional Neural Network
% This approach delivered significantly better results over multiple state-of-the-art architecture for face detection, alignment and landmark detection from the year 2016.
% \citep{Zhang:2016:MTCCN}
\newline\newline
Combining various signals by fusing them before feeding them into a Machine Learning model is a common trend in research on \gls{ER}. The most obvious choice is to combine audio with visual signals, as was done by \citet{Yan:2016:MultiClueFusion} and \citet{Hossain:2019:AudioVisualER}. \citet{Xing:2019:EEGAudioVisual} were combining their visual and audio signals with signals from an EEG analysis. All of them argue, that their approach achieves significantly better results than their comparison baseline.\newline
Even though it is currently very common that signal fusion happens with video and audio signals, \citet{Akcay:2020:SpeechEmotionRecognition(SER)} point out, that audio signals could be combined with a wide variety of signals, such as: visual signals, physiological signals, linguistic features, mouse movements and keystroke dynamics.


%%%%%%%%%%%%%%%%%%%%%%%%%%%%%%%%%%%%%%%%%%%%%%%%%%%%%%%%%%%%%%%%%%%%%%

\section{Identification of human intentions}
The here presented literature was gather because of the intention of applying the Emotion Recognition results obtained during this Master thesis in a real-life application. Therefore, research was done to lay out the state-of-the-art about the identification of human intentions by observation, primarily of facial expressions. The envisioned application might use Emotion Recognition to support consultants during video calls with customers.
\newline\newline
Especially interesting for such an application would be the identification of human interest. As research is really sparse in that specific area, all the current research related to the identification of human intentions will be presented in the following paragraphs:
\newline\newline
\citet{Dong:2012:UnderstandHumanImplicitIntention} did an experiment where they let participants agree or disagree towards a statement. At the same time, the participants were subjected to an \gls{EEG} analysis in order to measure brain activity. Thus, they were able to tell by a person's brain activity whether he/she is agreeing or disagreeing with a statement even before that person actually vocalized it. Even though the results are very convincing, such an approach would be hardly practical as an \gls{EEG} analysis cannot be quickly performed, especially not with In-The-Wild data.
\newline\newline
Another interesting approach was conducted by \citet{Esser:2018:LandmarkDetection} who used \gls{ER} for facial expression to visualize the experience of a patient's pain. He made us of an \gls{AAM} for feature extraction and hence, to better predict the Action Units of a person's face. As a result, the author could determine whether specific Action Units known for expressing pain are present in a face. 
\newline\newline
Furthermore, researchers were identifying customer satisfaction, as did \citet{Ren:2012:ERforCustomerSatisfaction} by analyzing customer words and comments through Emotion Recognition in order to predict emotions in six categorical classes. They authors underline the importance of customer satisfaction, however, they assume that the correct recognition of emotions can be directly translated into customer satisfaction by human interpretation. The goal should be to have an mechanism that automatically calculates such a measurement and gives back meaningful results.
% Paper from 2012 used customer words and comments to detect customer satisfaction through Emotion Recognition in this information:
% Using an annotated emotion corpus (Ren-CECps), we
% first present a general evaluation of customer satisfaction
% by comparing the linguistic characteristics of emotional
% expressions of positive and negative attitudes. The associations in four negative emotions are further investigated.
% After that, we build a fine-grained emotion recognition
% system based on machine learning algorithms for measuring customer satisfaction; it can detect and recognize
% multiple emotions using customers’ words or comments.
% The results indicate that blended emotion recognition is
% able to gain rich feedback data from customers, which can
% provide more appropriate follow-up for customer relationship management. \citep{Ren:2012:ERforCustomerSatisfaction}
\newline\newline
A different approach for the identification of customer satisfaction was conducted by \citet{Kamaruddin:2016:MeasuringCustomerSatisfaction}. They used did predict emotions from speech signals and based upon that they inferred customer satisfaction with a very simple approach. The approach hypothesis is that a customer is either satisfied if the recognized emotion has a positive value for valence (= a positive emotion) or unsatisfied if the value is negative.

\begin{figure}[H]
  \begin{center}
  \includegraphics[angle=0, width=0.6\textwidth]{Figures/Satisfaction_from_VA.PNG}
  \caption{Satisfaction inference from VA values}
  \label{fig:SatisfactionFromVA}
  \end{center}
\end{figure}

\citet{Kamaruddin:2016:MeasuringCustomerSatisfaction} realized that using a threshold for defining a 'Neutral' region between Satisfied and Unsatisfied actually performs better in terms of accuracy. They admit that their results are still weak with 40 \% accuracy for measuring satisfaction and 58 \% accuracy for neutral emotion.
% In another approach by \citet{Kamaruddin:2016:MeasuringCustomerSatisfaction}, they built upon the hypothesis that the customer is satisfied if he/she is experiencing happiness and neutral emotion whereas if he/she is experiencing sadness or anger, the customer is dissatisfied. They were using Valence and Arousal values and split them up into four emotional values, including one for neutral emotion. They made use of a threshold to define the neutral emotion class. From these classes they directly inferred whether the customer was satisfied or not. Their accuracy stems from the correct prediction of this classes and is with 39 percent accuracy much better than random guessing with 25 percent.
% \begin{quote}
%     We hypothesize that if the valence value is positive (happiness and calm), the customer is satisfied whereas if the valence value is negative (anger and sadness), the customer is not satisfied. Although such approach is simple, it may give us the better understanding of neutral region threshold so that it can be further used for analysis. \citep{Kamaruddin:2016:MeasuringCustomerSatisfaction}
% \end{quote}
\newline\newline
In a study on ads appreciation conducted by \citet{Poirier:2016:AdsFacialExpression}, the authors measured the effectiveness of ads by analyzing facial expressions. \citet{Poirier:2016:AdsFacialExpression} claims that the emotional journey is the strongest predictor of ad appreciation. Thus, they predict the emotional values for valence and arousal, and look at the curve/profile created by values for valence. The author's assumption seems to be that a certain profile of the valence curve determines whether an ad is successful. This sounds indeed very promising, but also raises the questions whether these ad appreciations in commercial ads also convert into actual positive ratings or even product purchases.
% \citet{Poirier:2016:AdsFacialExpression}, the authors measured the effectiveness of ads through facial expressions.
% \begin{quote}
%     emotional journey, which relates to the positive or negative emotional variation  (valence between -1 and 1 where -1: 100\% negative expression, 1: 100\% positive expression and 0: neutral expression), remains the most powerful predictor of ad appreciation 
% \end{quote}
\newline\newline
\citet{Yeasin:2006:MeasurmentOfInterestFromVideo} present an spatio-temporal approach where they categorize emotions in six classes based on visual data. Subsequently, they use those predicted classes, map them to a 3D affect space and compute the level of interest by calculating: L = W x I.
Hereby, the weight W stands for the relative number of images where most of a motion concentrates. While I represents the intensity of an emotion which is calculated by summing up the values for valence, arousal and stance (3D affect space).
% presents a spatio–temporal approach in recognizing six universal facial expressions from visual data and using them to compute levels of interest. 
% Recognized emotions and used them to calculate a level of interest L, which is calculated as follows: L = W x I. Where I stands for the intensity of an emotion and is calculated through summing up all values from valence, arousal and stance (in a 3D affect space) and mapping these to a range of -5 to +5.
% The weight W is represented by a number in the range of 0 to +1 and measures the relative number of images in a sequence that concentrate most of a motion. The lower the number, the closer the coefficient to 1. \citep{Yeasin:2006:MeasurmentOfInterestFromVideo}



% The FaceReader application \citep{Noldus:2020:Facereader} from Noldus bases their 'Interest' calculation on certain action units instead of recognized emotion values. These action units include:
% \begin{itemize}
%     \item 01 - Inner Brow Raiser
%     \item 02 - Outer Brow Raiser
%     \item 03 - Upper Lid Raiser
%     \item 17 - Chin Raiser
%     \item 20 - Lip Stretcher
%     \item 26 - Jaw Drop
% \end{itemize}
% For the analysis the following time interval is used: Interest - 2 seconds.

\chapter{Methodology}
% (diagram: data flow ; from input to output, everything in between; a paragraph on each module)


\section{Machine Learning Workflow}
The Machine Learning workflow utilized during this Master thesis is based upon a universal blueprint proposed by \citet{Chollet:2017:DeepLearningPython} and will lay out how the underlying Machine Learning problem got tackled. It consists of the following 7 steps which are displayed with a few keywords describing the approach proposed by this Master thesis:

\begin{enumerate}
    \item \textbf{Defining the problem and assembling a dataset}\newline
    Emotion Recognition, Regression problem, AFEW-VA dataset
    \item \textbf{Choosing a measure of success}\newline
    RMSE, CORR
    \item \textbf{Deciding on an evaluation protocol}\newline
    5-fold-Cross-Validation
    \item \textbf{Preparing your data}\newline
    Restructure frames, Resize range of label's value
    \item \textbf{Developing a model that does better than baseline}\newline
    Implementing a first version model
    \item \textbf{Scaling up: developing a model that overfits}\newline
    Implementing the pre-trained neural network ResNet50
    \item \textbf{Regularizing your model and tuning your hyperparameters}\newline
    Dropout, DataAugmentation
\end{enumerate}



\section{Research paradigm: Ablation study}

\begin{quote}
    An Ablation Study, in medical and psychological research, is a research method in which the roles and functions of an organ, tissue, or any part of a living organism, is examined through its surgical removal and observing the behaviour of the organism in its absence.\citep{Sheikholeslami:2019:AblationProgrammingML}
\end{quote}


Ablation Study in Machine Learning is derived from its medical/psychological background and can be defined as a scientific examination of Machine Learning systems by purposefully removing features or components, in order to observe its effects on the system's performance. Thus, every design choice or module can be included in an ablation study. As a result, Ablation Study can provide valuable insights to researchers, albeit it doesn't provide sufficient proof for drawing direct conclusion on the module's contribution. \citep{Sheikholeslami:2019:AblationProgrammingML}
\newline\newline
The idea behind ablation study applied to Artificial Neural Networks (ANNs) in this Master thesis is that the network is first trained to perform its Emotion Recognition task. Afterwards, modules or design choices are removed from the network which allows the researcher to measure the performance change due to the caused damage. This paradigm will be mostly used in chapter 5.4, where experiments on top of the already developed Neural Network will be conducted. \citep{Fadelli:2018:AblationInANN}


% \subsection{Development technique: Minimum Viable Product (MVP)}
% The concept of a Minimum Viable Product (MVP) was first introduced by \citet{Ries:2011:TheLeanStartUp} as a methodology for creating a 'Lean StartUp'. It aims at starting the learning process as early as possible through integrating early adopter's feedback.\citep{Lenarduzzi:2016:MVP}
% \newline\newline
% \citet{Lenarduzzi:2016:MVP} argues that the Build-Measure-Learn loop is one of the core principles in 'Lean StartUp' which allows entrepreneurs to learn whether to give up or persevere with a current build. This build is usually defined as a MVP. The author \citet{Ries:2011:TheLeanStartUp} describes an Minimum Viable Product as follows:
% \begin{quote}
%     "The MVP is that version of the product that enables a full turn of the Build-Measure-Learn loop with a minimum amount of effort and the least amount of development time." \citep{Ries:2011:TheLeanStartUp}
% \end{quote}
% However, \citet{Ries:2011:TheLeanStartUp} clarifies that a MVP is still far from being complete, on the contrary, it requires additional effort during building as its outcomes will be presented to potential customers and its feedback needs to be measurable. 
% \newline\newline
% The MVP for this this starts off with just the code for training a artificial neural network and testing its achievements on previously unseen data. This allowed the author to show a fully functioning product from the beginning on and to add functionality with the progress of the experiments.
% \newline\newline
% While the MVP is being expanded the measurable results are being presented continually to the supervisors from the University of Hamburg, as well as the supervisor from PPI AG. The resulting feedback flows back into the product by the design of the future experiments and changes. Thus, following the philosophy of the MVP, allowed the author to steer the direction of the experiments as needed and to continually receive actionable feedback on the current state of the product.

    
%%%%%%%%%%%%%%%%%%%%%%%%%%%%%%%%%%%%%%%%%%%%%%%%%%%%%


\section{Methods}	

The approach proposed in this Master thesis is comprised of multiple methods which are illustrated in the following figure \ref{fig:MachineLearningModelMethods}. On the left hand side of the figure, a summary of the approach is illustrated in a consecutive order from top to bottom. Each important step has a number assigned to it which can be found on the right hand side with a corresponding example picture of the output of this step.

\begin{figure}[H]
  \begin{center}
  \makebox[\textwidth][c]{\includegraphics[width=1.2\textwidth]{Figures/DataFlow_Diagram.png}}%
  %\includegraphics[angle=0, width=1.0\textwidth]{Figures/DataFlow_Diagram.png}
  \caption{Overview: Machine Learning Model - Methods}
  \label{fig:MachineLearningModelMethods}
  \end{center}
\end{figure}




\subsection{Preprocessing}
First of all, the data structure of the selected database needs to be analyzed and according to this structure data is being imported and structured. A list of all the filenames of the images and their respective labels for valence and arousal is created for each, the training and testing data. While the training data set is being shuffled for a better generalization during training, the testing data is used for validating the training results on a previously unseen part of the dataset.
\newline\newline
Based upon the list of filenames the actual image data is being read in and converted into RGB values. Furthermore, the values of valence and arousal are divided by 10, as the labels need to be brought into a range of -1 to 1 in order to fit the 'tanh' activation function of the model's final output layer.
\newline\newline
The labels and image data used for training are subsequently split into 80 \% training and 20 \% cross-validation data. 
% The last preprocessing step involves loading all the images, converting it into RGB values and then extracting the face while using the face detection techniques described in the next chapter. The cropped output image will have a shape of 224x224x3 and will be permanently saved so that this step doesn't have to be repeated each time the model is being trained. Furthermore, the model can then reach back to this array of extracted images and directly use them to train or finetune the model.

\begin{center}
\begin{figure}[H]
  \begin{center}
  \includegraphics[angle=0, width=0.5\textwidth]{Figures/method_1.png}
  \caption{Method 1}
  \label{fig:MachineLearningModelMethod_1}
  \end{center}
\end{figure}
\end{center}


\subsection{Face detection \& bounding box}
The \gls{MTCNN} poroposed by \citet{Zhang:2016:MTCCN} is a pre-trained neural network optimized for the tasks of simultaneous face detection, face alignment, bounding boxing and landmark detection.\citep{Brownlee:2019:VggFace2HowToFaceRec}
\newline\newline
In this Master thesis, the \gls{MTCNN} model was used to detect the face in an image and determine it's coordinates for the bounding box. Thus, after a new image was read in and converted into RGB color values, the \gls{MTCNN} module can detect the face and determine its bounding box. With the bounding box coordinates a rectangle is being drawn on the input image and saved for further processing.

\begin{center}
\begin{figure}[H]
  \begin{center}
  \includegraphics[angle=0, width=0.5\textwidth]{Figures/method_2.png}
  \caption{Method 2}
  \label{fig:MachineLearningModelMethod_2}
  \end{center}
\end{figure}
\end{center}

\subsection{Landmark detection \& Heatmap generation}
For the detection of landmarks the 'Face Landmark Detection' algorithm from the dlib library is implemented. This algorithm creates a shape model which it is aligning to the face at hand by refining its positions through a cascade of regressors. Additionally, dlib offers an already pre-trained version of their 'Face Landmark Detection' algorithm as a pre-trained shape predictor that can determine 68 facial landmarks.\citep{Kazemi:2014:ShapePredictor}
\newline\newline
For the implementation, the shape predictor requires, next to the input image, the bounding box of the person's face appearing in the input image. As the bounding box was already determined during the previous step by the \gls{MTCNN} module it can be directly used as an input for the landmark detection. \citep{Datahacker:2020:DlibFacialLandmarks}
\newline\newline

\begin{quote}
    imgaug is a library for image augmentation in machine learning experiments. It supports a wide range of augmentation techniques ... it can not only augment images, but also keypoints/landmarks, bounding boxes, heatmaps and segmentation maps. \citep{Jung:2020:Imgaug}
\end{quote}
For the proposed approach, imgaug was used to convert the previously detected landmarks into an heatmap and apply it as an overly onto the image.

\begin{center}
\begin{figure}[H]
  \begin{center}
  \includegraphics[angle=0, width=0.5\textwidth]{Figures/method_3.png}
  \caption{Method 3}
  \label{fig:MachineLearningModelMethod_3}
  \end{center}
\end{figure}
\end{center}

\subsection{Face extraction}
The input for this step consists of the image with a heatmap as an overlay and the coordinates for the face's bounding box. Thus, the face is simply cropped along the lines of the rectangle, comprised by the bounding boxe's coordinates. % \citep{Brownlee:2019:VggFace2HowToFaceRec}

\begin{center}
\begin{figure}[H]
  \begin{center}
  \includegraphics[angle=0, width=0.5\textwidth]{Figures/method_4.png}
  \caption{Method 4}
  \label{fig:MachineLearningModelMethod_4}
  \end{center}
\end{figure}
\end{center}

\subsection{Data Augmentation}
In order to make my model generalize better, Data Augmentation is being aplied beforehand to the pictures themselves using the ImageDataGenerator provided by Keras. It works as follows:
\begin{itemize}
    \item Taking a batch of images
    \item Apply random transformations on it
    \item Replace the original batch with the newly transformed batch
\end{itemize}
For this proposed approach data is being transformed randomly in terms of rotation, width shift, height shift, horizontal flip, brightness and zoom.

\begin{center}
\begin{figure}[H]
  \begin{center}
  \includegraphics[angle=0, width=0.5\textwidth]{Figures/method_5.png}
  \caption{Method 5}
  \label{fig:MachineLearningModelMethod_5}
  \end{center}
\end{figure}
\end{center}


\subsection{Machine Learning Model for Emotion Recognition}
In order to get a state-of-the-art Machine Learning model for Emotion Recognition, the author decided to make use of a pre-trained model that was already trained to solve a similar challenge and fine-tune it then. 
\newline\newline
The decision fell on the VGGFace Model which is optimized for face recognition and was pre-trained on a large-scale face dataset named VGGFace2. The dataset contains about 3.31 million images of 9131 subjects and poses additionally a great variety in pose, age, etc. The architecture used for the VGGFace Model is a ResNet-50 Convolutional Neural Network that was trained on the VGGFace2 dataset for the purpose of face recognition. When the VGGFace2 paper\citet{Cao:2018:VGGFace2} was published in \citeyear{Cao:2018:VGGFace2}, its performance exceeded the pevious state-of-the-art by a large margin. \citep{Cao:2018:VGGFace2}
\newline\newline
Due to the fact, that VGGFace is optimized for face recognition it had to learn similar facial features to what is needed by emotion recognition. Therefore, the VGGFace model's main selling point is that it already learned how to extract facial features from an image. However, the last Dense layers of the model cannot be reused as they are trimmed to learn the connection between facial features and the output of the face recognition challenge. As a result, the last Dense layers are replaced by custom made layers which consist of a Dropout layer, a single Dense layer with 1024 units, another Dropout layer, a Batch Normalization layer and two Dense layers, one for each output value.
\newline\newline
Using the above described architecture, finetuning needs to be conducted in order to fit the pre-trained model better to the Emotion Recognition challenge at hand. This means, that the weights of the pre-trained model are used as a basis for training on the AFEW-VA dataset with its images and respective labels.
\chapter{Implementation}
\section{Experimental Setup}
In order to conduct the training process described in the previous chapter, it was necessary to have a computer with a powerful GPU in order to speed up training. For this Master thesis, the experiments/training were conducted on a Nvidia Titan X (Pascal) GPU with 12 GB of memory. 
\newline\newline
For the ease of reproducibility of this work, frameworks and major libraries are hightlighted as follows: For machine learning, Keras 2.2.5 with TensorFlow 1.14.0 as a backend. In addition to that, Keras-VggFace 0.6, MTCNN 0.1.0 and OpenCV 4.1 were used. The entire code was implemented in a Python 3.6 environment.


\section{Dataset}
The selected Acted Facial Expressions in the wild - Valence Arousal (AFEW-VA) dataset introduced by \citet{Kossaifi:2017:AFEW-VADatabase} is based upon the Acted Facial Expressions in the wild database (AFEW) introduced in \citeyear{Dhall:2012:AFEW} by \citet{Dhall:2012:AFEW}. The AFEW dataset is composed of video clips that try to depict a real-world environment. It captures facial expressions, natural head pose movements, occlusions, subjects' races, gender, diverse ages, and multiple subjects in a scene. The authors labeled the video clips with one of six basic expressions: anger, disgust, fear, happiness, sadness, surprise, or neutral. The AFEW-VA dataset \citep{Kossaifi:2017:AFEW-VADatabase} uses the same underlying real-world video data, but did not annotate its video clips with one of six basic expressions. Instead it used the two dimensional affective space with valence and arousal. 
\newline\newline
Furthermore, as the name of the dataset already suggests, the dataset is compopsed of data that was collected under in-the-wild conditions. In-the-wild refers to real-life conditions in video clips, which make it significantly more challenging to recognize emotions in comparison to a dataset that has been captured in a controlled environment. \citet{Salah:2018:VideoBasedER} explained that these difficulties can be caused by, for example, uncontrolled illumination or uncontrolled video quality due to a different recording medium, like webcams by individuals vs. professional cameras. Such in-the-wild data is usually acquired from talk shows, movies or other natural interactions. 
\newline\newline
Since the research conducted during this Master's thesis is intended to serve as a basis for a further real-life application in video-call scenarios, it was clear that the dataset needed to be as close to in-the-wild conditions as possible. Furthermore, it was decided that it was more important to capture as much of an emotion's information as possible rather than placing value on the interpretation of emotions. As a result, the AFEW-VA dataset got chosen because of its in-the-wild conditions, its 2D Affective Space model for emotion representation and its backing by the scientific community when it comes to providing comparable results.
\newline\newline
The AFEW-VA dataset is made up of 600 video clips, each consisting of multiple frames that make up the video clip. Further statistics about the video-clips, frames and subjects of the dataset are presented in Table \ref{tab:DatasetStatistics}. Furthermore, each frame is then annotated in terms of valence and arousal. Both values have an annotation level ranging from -10 to +10 in full integer values. This results in a total of 21 levels.\citep{Kossaifi:2017:AFEW-VADatabase} 

\begin{table}[H]
\begin{center}
\begin{tabular}{@{}lc@{}}
\toprule
\textbf{AFEW-VA dataset statistics} &  \\ \midrule
total no. of video-clips & 600 \\
total no. of frames & 30051 \\
total no. of subjects & 240 \\
avg. no. of videos per subject & 2.5 \\
min. no. of frames per video & 10 \\
max. no. of frames per video & 145 \\
avg. no. of frames per video & 50.1 \\
median no. of frames per video & 45.5 \\ \bottomrule
\end{tabular}
\caption[AFEW-VA dataset statistics]{Gathered statistics about the no. of video-clips, frames and subjects in the AFEW-VA database.}
\label{tab:DatasetStatistics}
\end{center}
\end{table}

%%%%%%%%%%%%%%%%%%%%%%%%%%%%%%%%%%%%%%%%%%%%
\section{Training \& Regularization}
\subsection{Backbone Network}

\begin{figure}[H]
  \begin{center}
  \includegraphics[angle=0, width=1.0\textwidth]{Figures/resnet50.png}
  \citep{Dwivedi:2019:ResNetInKeras}
  \caption[ResNet50 architecture overview]{The original ResNet50 \citep{He:2015:DeepResidualLearningForImageRecognition} architecture is a very deep-layered architecture consisting of multiple layers inside each of the five stages.}
  \label{fig:ResNet50Architecture}
  \end{center}
\end{figure}

The chosen ResNet50 model \citep{He:2015:DeepResidualLearningForImageRecognition} consists of 50 layers, combined in 5 stages, and includes over 23 million trainable parameters. It was one of the very first deep-layered neural networks that introduced a solution to the notorious vanishing gradient problem by its 'skip connection' concept.
\newline\newline
'Skip connection' is done by adding a shortcut from the input to the output of a 'CONV' or 'ID' block, allowing the gradient to flow through. This behaviour, as illustrated in Figure \ref{fig:ResNet50ConvBlock}, makes sure that the subsequent block performs at least as well as the previous. An overview of the composition of all blocks in ResNet50 is presented in Figure \ref{fig:ResNet50Architecture}. 

\begin{figure}[H]
  \begin{center}
  \includegraphics[angle=0, width=0.9\textwidth]{Figures/ResNet50_ConvBlock.png}
  \citep{Dwivedi:2019:ResNetInKeras}
  \caption[ResNet50 skip connection]{In a Convolutional block of ResNet50 \citep{He:2015:DeepResidualLearningForImageRecognition}, additionally to adding the result, a convolution is performed on the original input.}
  \label{fig:ResNet50ConvBlock}
  \end{center}
\end{figure}


% ResNet50 is defined by a very deep-layered neural network which traditionally posed a difficult challenge to researchers, as the training process got worse the deeper the neural network due to the notorious vanishing gradient problem. The vanishing gradient leads to a rapid saturation of the model's weights which results in a degradation of the overall model performance. To fight this problem, ResNet introduced a novel concept, named skip connection, where the results of a block are added together with the original input before applying an activation function. \citep{Dwivedi:2019:ResNetInKeras}
% \newline\newline
% This concept is also illustrated in the following Figure \ref{fig:ResNet50IdentityBlock} for the identity block:

%\begin{figure}[H]
%  \begin{center}
%  \includegraphics[angle=0, width=0.9\textwidth]{Figures/ResNet50_IdentityBlock.png}
%   \caption{In an Identity block of ResNet50 \citep{Dwivedi:2019:ResNetInKeras}, a shortcut is added by adding the results to the original input.}
%   \label{fig:ResNet50IdentityBlock}
%   \end{center}
% \end{figure}

% However, this operation is only possible when the two inputs for the addition have the same shape. Due to the layout of the ResNet50 architecture, this is inherently the case for the Identity Block. For the Convolution Block a further operation is needed to transform the block's original input into the same shape as the outcome of the Convolutional layers. As can be seen in Figure \ref{fig:ResNet50ConvBlock}, this was done by applying a separate Convolutional plus Batch Normalization layer to the original input and choosing its hyperparameters in a way that the output will be in the same shape as the outcome of the Convolutional layers.

% \citet{Dwivedi:2019:ResNetInKeras} argued that this approach is helpful in mitigating the problem of vanishing gradients, as its skip connection concept provided an alternate shortcut path for gradients to flow through. This ensured that the subsequent block will perform at least as well as the previous.




%%%%%%%%%%%%%%%%%%%%%%%%%%%%%%%%%%%%

% \begin{figure}[H]
%   \begin{center}
%   \includegraphics[angle=0, width=1.0\textwidth]{Figures/model_architecture.PNG}
%   \caption{The neural network architecture is constructed by loading the pre-trained VGGFace model and adding a custom classifier.}
%   \label{fig:NNArchitecture}
%   \end{center}
% \end{figure}

% Figure \ref{fig:NNArchitecture} describes the architecture of the neural network with the following elements:

% \begin{itemize}
%     \item Line \#2: the pre-trained VGGFace model is being loaded with 'resnet50' as its neural network architecture. With the parameter 'include\_top' set to False allows to exclude the classifier of the pre-trained network on face detection.
%     \item Line \#5 to line \#10: The VGGFace model is either set to trainable or non-trainable. For the first three epochs of fune-tuning, the model gets set to non-trainable. Afterwards it is set to trainable, and allows the model to update its weights.
%     \item Line \#12: The output of the VGGFace model is being accessed and in line \#13 reduced in dimensionality in order to fit the input requirements for the following layer.
%     \item Line \#15 to line \#18 contains the classifier consisting of a Dense layer (1024 units), together with two Dropout and a Batch Normalization layer in order to improve generalization capabilities.
%     \item Line \#20 and \#21: The output for each evaluation metric is defined individually. This makes it possible to neatly visualize the outcomes separately. The 'tanh' activation function resized the output to a floating-point number in the range of -1 to +1.
% \end{itemize}

% The choice of using only a single Dense layer in the classifier was based on a comparison of fine-tuning strategies by \citep{Pittaras:2017:FineTuningStrategiesComparison}. Through this comparison of fine-tuning strategies on pre-trained neural networks, they could show that
% \begin{quote}
%     increasing the depth of a pre-trained network with one more fully-connected layer and fine-tuning the rest of the layers on the target dataset can improve the network’s concept detection accuracy compared to other fine-tuning approaches. \citep[~p. 103]{Pittaras:2017:FineTuningStrategiesComparison}
% \end{quote}

% Moreover, \citet{Pittaras:2017:FineTuningStrategiesComparison} could obtained their best results with a fine-tuning approach that replaced the pre-trained classifier of a network with a single Dense layer containing a high number of neurons.

%%%%%%%%%%%%%%%%%%%%%%%%%%%%%%%%%%%%%%%%%%%%%%%%%%%%%%%%%

\subsection{Training} \label{sec:Training&Regularization}
\subsubsection{Pre-trained network}
A pre-trained network is a saved neural network that has previously learned spatial hierarchy of features on a large and general dataset. As recommended by \citet{Chollet:2017:DeepLearningPython} using a pre-trained neural network is a highly effective approach when dealing with a small image dataset. With around 30.000 frames (= 600 video clips), the AFEW-VA dataset \citep{Kossaifi:2017:AFEW-VADatabase} can be considered as such. The utilized VGGFace network was pre-trained on the large-scale face recognition dataset VGGFace2 \citep{Cao:2018:VGGFace2}. This has the advantage, that face recognition poses similar challenges as found in emotion recognition, and by using the pre-trained VGGFace network, already learnt information can be reused for the emotion recognition challenge.

\subsubsection{Fine-tuning}
In order to be able to fine-tune a pre-trained network, it first had to be adapted to the current challenge. Therefore, in this work the classifier got replaced with a Fully-Connected plus two RNN layers. These RNN layers help to capture the tempo-spatial information between frames of a video-clip. The design decision for the classifier was based on previous experiments conducted by \citet{Kollias:2019:AffWild}.
% This decision was based on a comparison of fine-tuning strategies conducted by \citet{Pittaras:2017:FineTuningStrategiesComparison} who found out that they achieved the best results fine-tuning results when only adding a single Dense layer with a high number of neurons.
\newline\newline
Due to the random initialization of the custom added classifier, the first step when fine-tuning the model was to train the classifier for a few epochs. At the same time, all other layers from the pre-trained model were not allowed to train and adapt to the new challenge (=frozen). This prevents the error stemming from the random initialization to propagate through the whole network.\citep{Chollet:2017:DeepLearningPython}
\newline\newline
In the second fine-tuning step, the pre-trained neural network was allowed to train and adapt to the new task in accordance with the selected Multi-Phase Fine-Tuning (FT) \citep{Sarhan:2020:MultiPhaseFineTuning} strategy. In contrast to Single-Phase FT where network is trained by unfreezing a set amount of layers at once, in Multi-Phase FT the network is trained by successively unfreezing layers in phases. This is illustrated in Figure \ref{fig:MultiPhaseFT}. In each phase the network's training process is started again, but each time more layers are allowed to be trained and adapted to the new task. As a result, each phase can build on the progress made by the previous phase. According to \citet{Sarhan:2020:MultiPhaseFineTuning} Multi-Phase FT achieves greater accuracy in fewer training epochs in comparison to Single-Phase Fine-Tuning.

\begin{figure}[H]
  \begin{center}
  \includegraphics[angle=0, width=0.8\textwidth]{Figures/MultiPhaseFT.PNG}
  \caption[Multi-Phase Fine-Tuning overview]{Multi-Phase Fine-Tuning (FT) leads to greater accuracy in fewer training epochs in comparison to Single-Phase FT.}
  \label{fig:MultiPhaseFT}
  \end{center}
\end{figure}

%%%%%%%%%%%%%%%%%%%%%%%%%%%%%%%%%%%%%%%%%%%%%%%%%%%%%%%
\subsection{Regularization}
A central role in developing a successful solution with machine learning is making sure that an algorithm will perform well not only on the training data, but also on previously unseen data. In many machine learning challenges it is common, that a well performing algorithm on the training data performs bad on previously-unseen data. This happens when the model is trying to memorize the training dataset, instead of learning its underlying patterns. This behaviour is called 'overfitting'. Strategies explicitly designed for decreasing overfitting, even at the expense of increasing the training error, are known as regularization. \citep{Goodfellow:2016:DeepLearning}
\newline\newline
An intuitive way to prevent overfitting is to use more training data, as this exposes the model to more data that it needs to find a useful representation for. However, in reality gathering more data is often not a viable option, which is why further regularization techniques are necessary to force the model to focus on the most prominent patterns. The easiest way, according to \citet{Chollet:2017:DeepLearningPython}, is to reduce the neural network's size in terms of trainable parameters. This would result in a reduction of the model's learning capacity.

\subsubsection{Feature removal}
For the here proposed machine learning model it was not possible to increase the amount of training data, as this would make an objective comparison with benchmark paper impossible. Thus, further regularization techniques were applied. The first choice was also to reduce the network's size. The pre-trained VGGFace network already provides a big stack of layers that cannot be removed without losing valuable information. Therefore, the convolution network for the mask, as well as the shared layers at the end of the network were kept at a minimal size.
% Therefore, the custom layers were reduced to a single Dense layer, which was proposed in a similar way by \citet{Pittaras:2017:FineTuningStrategiesComparison}, 

\subsubsection{Dropout}
Dropout is a very commonly used regularization technique that introduces noise during training by randomly setting a certain percentage of the layer's output values to zero. The Dropout applied in the proposed architecture was determined through extensive experimentation and was applied before and after the single Fully-Connected (=Dense) layer with a rate of 0.5 and 0.5 respectively. The following Figure shows an example with a dropout rate of 0.5.

\begin{figure}[H]
  \begin{center}
  \includegraphics[angle=0, width=0.7\textwidth]{Figures/dropout.PNG}
  \caption[Dropout regularization]{The applied dropout sets a defined percentage share of the parameters of an activation matrix to zero.\citep{Chollet:2017:DeepLearningPython}}.
  \label{fig:Dropout}
  \end{center}
\end{figure}

Applying such a dropout rate allows a network to purposefully forget some information, which makes the learning process harder, but merits a better generalization.

\subsubsection{Data Augmentation}


%%%%%%%%%%%%%
\section{Data Augmentation}

\begin{figure}[H]
  \centering
  \subfloat[V: 0.0, A: +0.5]{\includegraphics[width=0.5\textwidth]{Figures/001/001_6_00000.png}}
  \hfill
  \subfloat[V: 0.0, A: +0.5]{\includegraphics[width=0.5\textwidth]{Figures/001/001_6_00011.png}}
  \hfill
  \subfloat[V: +0.2, A: +0.3]{\includegraphics[width=0.5\textwidth]{Figures/002/002_6_00000.png}}
  \hfill
  \subfloat[V: +0.2, A: +0.4]{\includegraphics[width=0.5\textwidth]{Figures/002/002_6_00011.png}}
  \hfill
  \subfloat[V: -0.5, A: +0.3]{\includegraphics[width=0.5\textwidth]{Figures/576/576_6_00000.png}}
  \hfill
  \subfloat[V: -0.6, A: +0.3]{\includegraphics[width=0.5\textwidth]{Figures/576/576_6_00011.png}}
  \caption[Data Augmentation]{Visualization of augmented faces and their corresponding labels, expressed by valence (V) and arousal (A) as values ranging from -1 (V: very negative; A: very calming) and +1 (V: very positive; A: very exciting)}
  \label{fig:MethodologyDataAugmentation}
\end{figure}

Data Augmentation is another technique that is used to increase noise during training by randomly transforming existing training samples into slightly different looking images. In that way, the model will see more slightly different images and increases its ability to better generalize from training data. Even though data augmentation was applied during training to both, the face and it's corresponding mask, Figure \ref{fig:MethodologyDataAugmentation} only presents examples from augmented faces.
\newline\newline
The Data Augmentation applied in the here proposed solution augmented the images randomly with the following parameters:

\begin{itemize}
    \item rotation range from 0 to 15 degrees
    \item width shift range from 0 to 15 percent of the total width
    \item height shift range from 0 to 15 percent of the total height
    \item horizontal flip
    \item brightness shift range from 80 to 120 percent
\end{itemize}

\subsubsection{Hyper-parameter optimization} \label{sec:HyperParameterOptimization}
Optimization of important hyper-parameters also greatly reduces the effects of overfitting. Two of the most impactful parameters are the learning rate and the batch size.
\newline\newline
An analysis conducted by \citet{Yuanzhi:2019:RegularizationInitialLargeLearningRate} on Initial Learning Rates confirmed that an initially large learning rate can have a regularization effect on the training process. Even though a small initial learning rate might allow for better performance initially, it will not be able to generalize as well as initially large learning rates. The training performed in this Master's thesis made use of an initial learning rate of 0.0001 for the first 3 epochs. Afterwards the training was also started off with a rate of 0.0001, but continuously decreased with each epoch by a factor of 0.95 through a learning rate scheduler.
\newline\newline
Along with the effects of the initial learning rate on improving generalization and reducing the effects of overfitting, \citet{Keskar:2016:LargeBatchTrainingGeneralization} posited that choosing small-batch methods consistently generalize better than large batch methods. Due to memory constraints it was not possible to experiment with large batch sizes, which is why for the here proposed approach a batch size of 2 was chosen. This translates into each batch consisting of two times the sequence length of 45.

%%%%%%%%%%%%%%%%%%%%%%%%%%%%%%%%%%%%%%%%%%%%%%%%%%%%%%%%%

\section{Evaluation \& Metrics}
\subsection{Evaluation} \label{sec:TrainValTestSplit}
Since the AFEW-VA dataset has no official validation and testing set, the AFEW-VA paper proposed by \citet{Kossaifi:2017:AFEW-VADatabase} and the simultaneous VA prediction paper proposed by \citet{Handrich:2020:SimultaneousPredVA} evaluate their models through splitting the AFEW-VA dataset into five disjoint, subject-independent folds. Subsequently, they performed a 5-fold-cross-validation for the prediction of valence and arousal values. 
\newline\newline
Due to time, complexity and resource constraints, it was chosen that this Master's thesis would not use the 5-fold-cross-validation evaluation approach. Instead, it makes use of the more prevalent method for machine learning model evaluation, namely the split of the dataset into a training, validation and testing subset. The training and validation subset were used during training to determine the optimal hyper-parameter settings. As these settings influence the performance on the validation subset, a third, independent subset was needed: the testing subset. This subset was used for a final performance evaluation after the optimal hyper-parameter were found.
\newline\newline
The following Figure \ref{fig:TrainValTestSplit} illustrates the chosen evaluation approach:

\begin{figure}[H]
  \begin{center}
  \includegraphics[angle=0, width=0.7\textwidth]{Figures/TrainValTestSplit.png}
  \caption[Dataset training-validation-test split]{The dataset was split into a training, validation and testing subset. While training and validation were used for the determination of optimal hyper-parameter settings, the testing subset was set aside for a final evaluation of the best performing model.}
  \label{fig:TrainValTestSplit}
  \end{center}
\end{figure}

% \begin{figure}[H]
%   \begin{center}
%   \includegraphics[angle=0, width=0.7\textwidth]{Figures/TrainTestSplit.png}
%   \caption{For the determination of optimal hyper-parameter settings, the dataset was split into a training, validation and testing subset.}
%   \label{fig:TrainTestSplit}
%   \end{center}
% \end{figure}

% As soon as optimal hyper-parameter settings were found during the training of the model, a 5-fold-cross-validation was conducted in order to make results comparable with the benchmark. For this 5-fold-cross-validation, previously determined optimal hyper-parameter settings were utilized for training and testing the model five times. As illustrated in Figure \ref{fig:CrossValidationSplit}, each time the model is trained during cross-validation, a different combination of folds was selected as training and testing subset.\newline

% \begin{figure}[H]
%   \begin{center}
%   \includegraphics[angle=0, width=0.7\textwidth]{Figures/CrossValSplit.png}
%   \caption{The dataset is split into five equal and subject-independent folds. Cross-Validation is performed by alternating the folds for training, validation and testing.}
%   \label{fig:CrossValidationSplit}
%   \end{center}
% \end{figure}

\subsection{Metrics} \label{sec:Metrics}
Considering the nature of the problem at hand, predicting the values for valence and arousal is better solved through regression, hence a metric such as accuracy, albeit common, will not provide useful information to reflect the performance of the system. Therefore, we stick to the measures used by \citet{Kossaifi:2017:AFEW-VADatabase}, namely root-mean-square error (RMSE) and Pearson product-moment correlation coefficient (CORR). 
\newline\newline
RMSE (eq. \ref{eq:RMSE}) is useful in giving the observer a notion of how close the predicted values are to the actual values, while CORR (eq. \ref{eq:CORR}) tells how strong the relationship between the prediction and the actual label is \citep{2020:RMSE} \citep{2020:PearsonCorrelation}.
  
\begin{equation} \label{eq:RMSE}
RMSE = \sqrt{(\frac{1}{n})\sum_{i=1}^{n}(y_{i} - x_{i})^{2}}
\end{equation}
\newline\newline
\begin{equation} \label{eq:CORR}
CORR = \frac{{}\sum_{i=1}^{n} (x_i - \overline{x})(y_i - \overline{y})}
{\sqrt{\sum_{i=1}^{n} (x_i - \overline{x})^2(y_i - \overline{y})^2}}
\end{equation}

where
\begin{conditions*}
 x_i  &  the actual ground truth value\\
 y_i  &  the predicted output value \\
 \overline{x}  &  the mean of all ground truth values \\
 \overline{y}  &  the mean of all predicted output values
\end{conditions*}

The RMSE is additionally used as the loss function for the optimization during training.

\subsection{Observation}
% Early Stopping & Model Checkpoint?
In order to find optimal hyper-parameter settings during training, the aforementioned metrics were continuously observed through an early stopping callback, as well as a model checkpoint callback. These callbacks made sure that the last best model, in terms of validation loss, was automatically backed up and that training was stopped as soon as the model successively failed to improve. The callbacks for this work are clearly defined in Figure \ref{fig:EarlyStopping} and \ref{fig:ModelCheckpoint}.

\begin{figure}[H]
  \begin{center}
  \includegraphics[angle=0, width=0.8\textwidth]{Figures/EarlyStopping.PNG}
  \caption[EarlyStoping callback]{The early stopping callback stops the training process as soon as there has not been any learning progress after a set period of epochs.}
  \label{fig:EarlyStopping}
  \end{center}
\end{figure}

\begin{figure}[H]
  \begin{center}
  \includegraphics[angle=0, width=0.8\textwidth]{Figures/ModelCheckpoint.PNG}
  \caption[Model checkpoint callback]{The model checkpoint callback tracks the best performing epoch according to a defined metric and automatically saves the weights of the best model.}
  \label{fig:ModelCheckpoint}
  \end{center}
\end{figure}

In summary, these callbacks reduce overfitting by stopping the learning process when no improvement on validation data was detected. Additionally, they allow to objectively compare different hyperparameter-settings of a model's architecture with one another.

\chapter{Results and Analysis}
% (more tables and graphs; comparison; what other people achieved)
The following results and experiments are based upon the AFEW-VA dataset and results will be compared along the metrics Root-Mean-Squared-Error (RMSE) and Pearson Correlation (CORR). The optimization goal is, to reduce the RMSE error, while improving the CORR as much as possible.

\section{Benchmark AFEW-VA Dataset}
In the original paper from 2017, when the AFEW-VA database \citep{Kossaifi:2017:AFEW-VADatabase} was introduced, \citet{Kossaifi:2017:AFEW-VADatabase} already provided a benchmark by comparing different methods, like SVR or DCNN, with the three metrics RMSE (Root Mean Squared Error), CORR (Correlation) and ICC (Interclass Correlation). In the following table, the best performing Deep Neural Network approach is compared with the overall best performing algorithm, called Multiple Kernel Learning (MKL):

\begin{table}[H]
\centering
\begin{tabular}{cc|ccc}
\multicolumn{1}{l}{} & \multicolumn{1}{l|}{\textbf{}} & \multicolumn{1}{l}{\cellcolor[HTML]{CBCEFB}\textbf{RMSE}} & \multicolumn{1}{l}{\cellcolor[HTML]{CBCEFB}\textbf{CORR}} & \multicolumn{1}{l}{\cellcolor[HTML]{CBCEFB}\textbf{ICC}} \\ \hline
FT-DCNN & \cellcolor[HTML]{F8FF00}\textbf{Valence} & 0.37 & 0.26 & - \\
(RGB images) & \cellcolor[HTML]{F8FF00}\textbf{Arousal} & 0.39 & 0.31 & - \\ \hline
MKL & \cellcolor[HTML]{F8FF00}\textbf{Valence} & 0.2639 & 0.401 & 0.274 \\
(Shape+DCT) & \cellcolor[HTML]{F8FF00}\textbf{Arousal} & 0.2229 & 0.445 & 0.34
\end{tabular}
\end{table}

The best performing method compared in this paper is the MKL which performs best in comparison to other methods examined in this paper. In this method the authors utilized different kernels for each, shape features (Norm-shape) and appearance features (Hybrid-DCT).
\newline\newline
The FT-DCNN model, a fine tuned Deep Convolutional Neural Network follows a very similar approach as the one proposed here in this Master thesis. The model is trained on randomly sampled frames from video sequences (just like in this thesis). Fine tuning is being done with AlexNet, a pre trained model on the ImageNet dataset.
\newline\newline
\citet{Theagarajan:2018:DeepDriver} in their 'DeepDriver' paper used the AFEW-VA database to evaluate their approach. It consisted of taking multiple frames as a sequence and feeding them into either a CNN-only architecture, as well as an CNN + LSTM architecture. Both methods heavily outperformed all benchmark results on the RMSE and CORR metric. The best result, using the CNN + LSTM, achieved the following results:


\begin{table}[H]
\centering
\begin{tabular}{c|cc}
\multicolumn{1}{l|}{\textbf{}} & \multicolumn{1}{l}{\cellcolor[HTML]{CBCEFB}\textbf{RMSE}} & \multicolumn{1}{l}{\cellcolor[HTML]{CBCEFB}\textbf{CORR}} \\ \hline
\cellcolor[HTML]{F8FF00}\textbf{Valence} & 0.093 & 0.639 \\ \hline
\cellcolor[HTML]{F8FF00}\textbf{Arousal} & 0.087 & 0.626 \\ 
\end{tabular}
\end{table}

These results were obtained using a 3-fold cross validation approach for training and evaluation. However, the author utilized two different datasets, the AFEW-VA and MotorTrend's dataset. Therefore, there training is being conducted on two separate datsets, which makes this approach not objectively comparable to the approach proposed in this Master thesis. 
\newline\newline
\citet{Handrich:2020:SimultaneousPredVA} made use of cross-database validation for the recognition of valence and arousal in videos/images. They used the AFEW-VA database \citep{Kossaifi:2017:AFEW-VADatabase} as a validation dataset and achieved much better results for the CORR and ICC metrics. This shows that their approach is indeed an improvement to the 2017 benchmark paper. Their proposed CNN architecture is also based upon RGB images as an input and achieves the following results:

\begin{table}[H]
\centering
\begin{tabular}{cl|cc}
\multicolumn{1}{l}{} & \textbf{} & \multicolumn{1}{l}{\cellcolor[HTML]{CBCEFB}\textbf{RMSE}} & \multicolumn{1}{l}{\cellcolor[HTML]{CBCEFB}\textbf{CORR}} \\ \hline
\begin{tabular}[c]{@{}c@{}}AFEW-VA\\ database\end{tabular} & \multicolumn{1}{c|}{\cellcolor[HTML]{F8FF00}\textbf{Valence}} & 0.26 & 0.39 \\
only & \multicolumn{1}{c|}{\cellcolor[HTML]{F8FF00}\textbf{Arousal}} & 0.25 & 0.29 \\ \hline
\begin{tabular}[c]{@{}c@{}}Cross-\\ database\end{tabular} & \cellcolor[HTML]{F8FF00}\textbf{Valence} & 0.28 & 0.58 \\
validation & \cellcolor[HTML]{F8FF00}\textbf{Arousal} & 0.26 & 0.46
\end{tabular}
\end{table}

These results were obtained using 5-fold cross validation on the AFEW-VA database, while training on 70 percent and validating on 30 percent of the whole dataset. Thus, the results are comparable to the herein mentioned approach. Looking closer at their results, an interesting behavior is visible: While the CORR metric increased substantially when using the Cross-database validation, the RMSE metric on the other hand did slightly worse than the training on the AFEW-VA database only.
\newline\newline
The following graph illustrates the results from all the methods tested by \citet{Kossaifi:2017:AFEW-VADatabase} together with the results achieved by \citet{Handrich:2020:SimultaneousPredVA}. The results from \citet{Handrich:2020:SimultaneousPredVA} are marked as 'Pr. CNN', as this graph is taken out of their paper:

\begin{figure}[H]
  \begin{center}
  \includegraphics[angle=0, width=0.9\textwidth]{Figures/benchmark.PNG}
  \caption{Benchmark from the paper 'Simultaneous Prediction of Valence/Arousal and Emotions on AffectNet, Aff-Wild and AFEW-VA' \citep{Handrich:2020:SimultaneousPredVA}}
  \label{fig:BenchmarkOnAFEW-VA}
  \end{center}
\end{figure}

It can be clearly seen that the proposed CNN approach by \citet{Handrich:2020:SimultaneousPredVA} easily outperforms the results of Neural Network based approaches (DCNN and FT-DCNN) by \citet{Kossaifi:2017:AFEW-VADatabase}. Only when comparing the results to the best performing algorithm, Multiple Kernel Learning (MKL), in terms of RMSE it reveals a slightly higher loss.


%%%%%%%%%%%%%%%%%%%%%%%%%%%%%%%%%%%%%%%%%%%%%%%%%%%%%%%%%%%

\section{Results}
The results for the approach proposed in this Master thesis were achieved by shuffling data while making sure that training, validation and testing data stays subject-independent. 
\newline\newline
The following two graphs display the learning curves during training for the selected metrics, namely RMSE and CORR, for the predicted Valence values. Subsequently, the respective learning curves are presented for the predicted Arousal values.

\hl{Learning Curve RMSE - Valence}

% \begin{figure}[H]
%   \begin{center}
%   \includegraphics[angle=0, width=0.9\textwidth]{Figures/xx.png}
%   \caption{RMSE metric - Learning curve for Valence}
%   \label{fig:RMSEValence}
%   \end{center}
% \end{figure}

\hl{Learning Curve CORR - Valence}

% \begin{figure}[H]
%   \begin{center}
%   \includegraphics[angle=0, width=0.9\textwidth]{Figures/xx.png}
%   \caption{CORR metric - Learning curve for Valence}
%   \label{fig:CORRValence}
%   \end{center}
% \end{figure}

\hl{Learning Curve RMSE - Arousal}
% \begin{figure}[H]
%   \begin{center}
%   \includegraphics[angle=0, width=0.9\textwidth]{Figures/xx.png}
%   \caption{RMSE metric - Learning curve for Arousal}
%   \label{fig:RMSEArousal}
%   \end{center}
% \end{figure}

\hl{Learning Curve CORR - Arousal}
% \begin{figure}[H]
%   \begin{center}
%   \includegraphics[angle=0, width=0.9\textwidth]{Figures/xx.png}
%   \caption{CORR metric - Learning curve for Arousal}
%   \label{fig:CORRArousal}
%   \end{center}
% \end{figure}


All these graphs show a similar behaviour: While the training curve looks optimal, the validation curve only improves slightly and fails to align with the training curve. Therefore, these results might indicate that the trained model is Overfitting on the training data. However, different architecture and hyper-parameter settings did not lead to any improvements in the evaluation on the testing data.
\newline\newline
The final results were achieved by using the design choices presented in this Master thesis and conduct a 5-fold cross-validation with subject-independent data. The following results represent the average of the results obtained during each of the 5-fold cross-validation:

\begin{table}[H]
\centering
\begin{tabular}{clll}
\multicolumn{1}{l}{\textbf{}} & \cellcolor[HTML]{9aff99}\textbf{ACC} & \cellcolor[HTML]{9aff99}\textbf{RMSE} & \cellcolor[HTML]{9aff99}\textbf{CORR} \\
\cellcolor[HTML]{F8FF00}\textbf{\begin{tabular}[c]{@{}c@{}}Valence\end{tabular}} & 0.28 & 0.27 & 0.22 \\ \hline
\cellcolor[HTML]{F8FF00}\textbf{\begin{tabular}[c]{@{}c@{}}Arousal\end{tabular}} & 0.10 & 0.25 & 0.27
\end{tabular}
\end{table}



%%%%%%%%%%%%%%%%%%%%%%%%%%%%%%%%%%%%%%%%%%%%%%%%%%%%%%%%%%%%%
\section{Comparison}
The following table gives an overview of the results obtained by \citet{Kossaifi:2017:AFEW-VADatabase},  \citet{Handrich:2020:SimultaneousPredVA} and by the approach proposed in this thesis. All the results are directly comprabale as they where achieved by performing the a 5-fold cross-validation with subject independent data between folds.

\begin{table}[H]
\begin{center}
\begin{tabular}{|cc|cc|cc|}
\hline
\rowcolor[HTML]{FCFF2F} 
\multicolumn{1}{|l}{\cellcolor[HTML]{FCFF2F}} & \textbf{Method} &  \textbf{\begin{tabular}[c]{@{}c@{}}RMSE\\ Valence\end{tabular}} & \textbf{\begin{tabular}[c]{@{}c@{}}CORR\\ Valence\end{tabular}} & \textbf{\begin{tabular}[c]{@{}c@{}}RMSE\\ Arousal\end{tabular}} & \textbf{\begin{tabular}[c]{@{}c@{}}CORR\\ Arousal\end{tabular}} \\ 
\hline
\textbf{\begin{tabular}[c]{@{}c@{}}RESULTS\\ THESIS\end{tabular}} & \begin{tabular}[c]{@{}c@{}}FT-DCNN\\ (VGGFace)\end{tabular} & 0.27 & 0.22 & 0.25 & 0.27 \\
\textbf{\begin{tabular}[c]{@{}c@{}} \\ AFEW-VA \\ \citep{Kossaifi:2017:AFEW-VADatabase} \end{tabular}} &
\begin{tabular}[c]{@{}c@{}}FT-DCNN\end{tabular} & 0.37 & 0.26 & 0.39 & 0.31 \\
\textbf{\begin{tabular}[c]{@{}c@{}} \\ Sim. prediction of VA \\ \citep{Handrich:2020:SimultaneousPredVA}\end{tabular}} & CNN & 0.26 & 0.39 & 0.25 & 0.29 \\ \hline
\end{tabular}
\end{center}
\end{table}

The results clearly show that the approach proposed in this Master thesis is substantially outperforming the results obtained by \citet{Kossaifi:2017:AFEW-VADatabase} in terms of Root-Mean-Squared-Error loss. It shows that RMSE is lower by 0.1 for valence and 0.14 for arousal, while the Correlation is lower by 0.04 for valence and 0.04 for arousal. In comparison to the approach proposed by \citet{Handrich:2020:SimultaneousPredVA} the thesis's results were pretty similar for RMSE, but worse for the CORR with 0.17 lower for valence and 0.02 lower for arousal.
\newline\newline
In conclusion, it can be said that the results of this thesis are up to date with the state-of-the-art proposed by \citet{Handrich:2020:SimultaneousPredVA} in \citeyear{Handrich:2020:SimultaneousPredVA} for Emotion Recognition In-The-Wild with the AFEW-VA dataset \citep{Kossaifi:2017:AFEW-VADatabase}. The results also show that fine-tuning a state-of-the-art pre-trained Neural Network is a very decicive factor in beating current state-of-the-art Machine Learning challenges. Furthermore, it is very likely that with further optimization, the proposed approach will even outperform the state-of-the-art.

\subsubsection{FaceReader application}
The FaceReader product of the company Noldus\citep{Noldus:2020:Facereader} is an application that is optimized for researchers to recognize emotions of participants. They make use of the categorical approach, but also convert them into values for valence and arousal. However, their approach is designed for laboratory conditions, thus they demand that the camera is placed slightly below eye-level and that there is no shadow in the person's face. Therefore, there claimed accuracy with 98 percent cannot be compared to In-The-Wild data. As confirmed through an experiment conducted by the author of this thesis, the FaceReader application failed to detect faces in every third (3 out of 9 seconds) due to imperfect conditions with a not perfectly illuminated room.
% However, as mentioned in the Quick Setup Guide, there approach is based upon the premise that the camera is placed slightly below eye-level of the test person and that good lightning, without shadow in the test person's face is crucial. Therefore, there claimed accuracy (from a phone call) with 98 percent cannot be compared to In-The-Wild data with imperfect conditions.
% Through a trial period the product was tested. As a camera an off-the-shelf integrated laptop camera was used in an weakly illuminated room. During the trial period the author realized, that the FaceReader often failed to provide information on expression intensity due to mostly not being able to find a face or also due to bad image quality. In total, in a 9 second experiment it was only able to detect the emotions during 3 of those 9 seconds.


%%%%%%%%%%%%%%%%%%%%%%%%%%%%%%%%%%%%%%%%%%%%%%%%%%%%%%%%%%%%%%%%
\section{Experiments}
Experiments are done with the approach of 'Ablation study'. This means each experiment analyzes a part of the implementation through leaving it out and comparing the results with the former proposed model.\newline
Basis for this is the 5-fold cross-validation approach with subject independent testing and validation data, as it was done by the benchmark papers.

%%%%%%%%%%%%%%%%%%%%%%%%%%%%%%%%%%%%%%%%%%%%%%%%%%%%%%%%%%%%%%%%
\subsection{Experiment 1: MTCNN (Multi-task Cascaded Convolutional Neural Network)}
\textbf{- How often does MTCNN fail to detect faces??}\newline
Out of 30051 frames it only failed to detect a face in 961 faces, this presents 3.2 percent of all images.
\newline\newline
\textbf{- What happens when not using MTCNN and directly feeding the original images into the VGGFace model for fine tuning.}\newline
The model performance slightly increased

\subsection{Experiment 8: Frames where no image is being detected}
So far feeding in the original image, in case MTCNN can't detect a face
-> what happens when deleting the image and not including it in the training data

\subsubsection{Optional: With MTCNN , but instead of keeping the Originalwhen no face was detected -> instead discarding this pair of (image, label)}

%%%%%%%%%%%%%%%%%%%%%%%%%%%%%%%%%%%%%%%%%%%%%%%%%%%%%%%%%%%%%%%%
\subsection{Experiment 2: Data Augmentation}
 In comparison to the results mentioned above, adding data augmentation couldn't show any improvement in performance.
\newline\newline
This data augmentation was applied:
-
\newline\newline
These are the results for the rmse metric for valance and arousal:
No improvement detected

%%%%%%%%%%%%%%%%%%%%%%%%%%%%%%%%%%%%%%%%%%%%%%%%%%%%%%%%%%%%%%%%
\subsection{Experiment 3: Regularization}
\textbf{Which regularization techniques can be applied to provide further improvements on the model performance in terms of loss reduction?}
\textbf{- Dropout layer}
\newline\newline
\textbf{- BatchNormalization layer}\newline
- BatchNormalization layer proved to provide constantly significant better results.
\newline\newline
\textbf{- Model architecture}\newline
- Despite \citet{Pittaras:2017:FineTuningStrategiesComparison} arguing that generally the best approach for finetuning is to only add one Dense Layer with a high number of neurons to the original pretrained model, the obtained results weren't automatically performing better, mostly worse.
\newline\newline
\textbf{- L2 regularization on pretrained model}\newline
Validation loss during training is lower than the training loss, which is very likely because of the reason that regularization was applied. This L2 regularization is only applied during training, but not validation/testing, which explains the significant difference between those values.
https://www.pyimagesearch.com/2019/10/14/why-is-my-validation-loss-lower-than-my-training-loss/

%%%%%%%%%%%%%%%%%%%%%%%%%%%%%%%%%%%%%%%%%%%%%%%%%%%%%%%%%%%%%%%%
\subsection{Experiment 4: LSTM}
\textbf{- Can an LSTM capture the time-spatio changes between frames and thus enhance the performance of \gls{ER}}
Data needs to be non-shuffled in order to do that
-> Performance didn't increase



%%%%%%%%%%%%%%%%%%%%%%%%%%%%%%%%%%%%%%%%%%%%%%%%%%%%%%%%%%%%%%%%
\subsection{Experiment 5: Landmarks from ASM (Active Shape Model)}
\textbf{- Can an AAM decrease the loss of the model performance?}
\newline\newline
\textbf{- How much longer does the program need to run through for one image with an AAM?}
\newline\newline
Construction of landmarks for the whole dataset of 30051 images. From 7402 images it could not detect the face (with dlib face detector)
\newline\newline
As described by \citet{Gao:2010:ActiveAppearanceModels} in an a summarizing paper, called 'Overview of Active Appearance Models', this approach is very suitable for the extraction of compact features in various application. However, they admit that this approach can't satisfy real-time requirements sufficiently, due to its time consuming fitting process.
\newline\newline
This consideration let to the decision to first implement a fast performing algorithm for the task of landmark prediction, instead of an Active Appearance Model which needs for each face a few seconds to be fitted. The decision fell on the algorithm of 'Ensemble of Regression Trees(ERT)' \citep{Kazemi:2014:ShapePredictor} implemented as a pre-trained shape predictor by the dlib library.
\newline\newline
Pre-trained shape predictor from dlib library, proposed by \citet{Kazemi:2014:ShapePredictor} in order to predict 68 facial landmarks. The author also argues that the approach 'Ensemble of Regression Trees (ERT)' is better than Active Shape Model approaches. A comparison between two variants of Active Shape Models the proposed ERT approach show that ERT cuts in half the error of the Active Shape Model variants.
\newline\newline
The 'default' face detector from dlib fails to detect 7402 out of 30051 images, while the MTCNN only fails in 961 images to detect a face. => A logical conclusion would be to use MTCNN to detect the face and the respective bounding box while using the dlib shape predictor, based upon the ERT approach, to time-efficiently detect  facial landmarks.
\newline\newline
\subsubsection{Landmarks only}
When feeding only landmarks into a model with Dense layers there was no improvement.
Even when the units of the Dense layers were drastically increased there was no improvement in terms of training, as well as validation loss. As a result, this experiment showed that with the coordinates of landmarks alone, the Neural Network is not able to predict emotions.

\subsubsection{Combination of images and landmarks}
\textbf{Performance improvement}


%%%%%%%%%%%%%%%%%%%%%%%%%%%%%%%%%%%%%%%%%%%%%%%%%%%%%%%%%%%%%%%%
\subsection{Experiment 6: Heatmap from ASM (Active Shape Model)}

\textbf{Comparison ASM: Landmarks vs. Heatmap}
- considerations run-time per image/batch
- validation/testing loss improvement

%%%%%%%%%%%%%%%%%%%%%%%%%%%%%%%%%%%%%%%%%%%%%%%%%%%%%%%%%%%%%%%%
\subsection{Experiment 7: Heatmap from AAM (Active Appearance Model)}
\textbf{Comparison ASM vs. AAM}
- considerations run-time per image/batch
- validation/testing loss improvement



\chapter{Application}
The goal for PPI AG with the Emotion Recognition approach proposed in this thesis, is to use it a scientific basis for an application in their business area. Thus, the idea is to assess in this chapter whether such an real-life application for the assistance of consultants during video-calls is viable. In order to provide consultants insightful assistance, it is necessary to go beyond the mere emotions and provide a meaningful metric. It was chosen to use interest as the metric for identifying the intention of a customer.
\newline\newline
In order to assess the viability of such an application in real-time video calls, it is necessary to develop an identification approach based upon current literature, design an experiment and compare the achieved results. A detailed explanation of these points will be provided in the following sections, starting with the identification approach.

\section{Identification approach}
%Is it viable to use Emotion Recognition to analyse consultations during video call in real-time in order to draw conclusions about human intentions (e.g. interest)?
% Unlike in the FaceReader application which bases their Interest prediction on action units, the selected approach is based upon an idea from a 2016 paper \citep{Kamaruddin:2016:MeasuringCustomerSatisfaction} The idea is to determine the level of interest/appreciation only through considering how positive or negative the average value of valence is.
% \newline\newline
% Furthermore, the authors classified their VA-values into four emotion categories and made use of a threshold to determine a neutral emotion category. Inspired by this approach, a threshold will be used to better identify whether a person is interested, neutral, or uninterested.
% \newline\newline
% However interesting from the FaceReader application \citep{Noldus:2020:Facereader} is, that they use an interval o f2 seconds to calculate their value of 'Interest'. This means that the currently calculated value for interest is based on video-frames from the last two seconds.

% This Master thesis is based on the utilized data set and thus is taking over the underlying assumption that facial expressions equate emotions. 
% However, in this application scenario new data is utilized that is not based upon this assumption. As the label, an actual purchase, doesn't care about that.

\section{Experimental setup \& results}
Goal is to compare the test person's perceived subjective interest of a (sales) video-clip with the interest predicted by the before mentioned approach from the web-cam video. Thus, an experiment needs to be set up that allows to record a video from the test person's webcam. Furthermore, a mechanism for obtaining the test person's feedback in terms of interest needs to devised.
\newline\newline
Considerations about different experiment setups are presented in the following table. Hereby, an 'X' indicates the non-fulfillment of an indispensable quality of an approach, while '/' indicates a constraint and ':)' indicates that the quality is fully fulfilled by the approach.
% Please add the following required packages to your document preamble:
% \usepackage[table,xcdraw]{xcolor}
% If you use beamer only pass "xcolor=table" option, i.e. \documentclass[xcolor=table]{beamer}
\begin{table}[H]
\begin{tabular}{ccccc}
\hline
\textbf{} & \begin{tabular}[c]{@{}c@{}}Web-Cam\\ Video\end{tabular} & \begin{tabular}[c]{@{}c@{}}Continuous\\ interest rating\end{tabular} & \begin{tabular}[c]{@{}c@{}}Easily\\ deployable\end{tabular} & \begin{tabular}[c]{@{}c@{}}Results\\ retrievable\end{tabular} \\ \hline
\textbf{\begin{tabular}[c]{@{}c@{}}Questionnaire\\ (e.g. Google Forms)\end{tabular}} & {\color[HTML]{FE0000} \textbf{X}} & {\color[HTML]{FE0000} \textbf{X}} & {\color[HTML]{32CB00} \textbf{:)}} & {\color[HTML]{32CB00} \textbf{:)}} \\
\textbf{\begin{tabular}[c]{@{}c@{}}Packaged Python\\ Application\end{tabular}} & {\color[HTML]{32CB00} \textbf{:)}} & {\color[HTML]{32CB00} \textbf{:)}} & {\color[HTML]{FE0000} \textbf{X}} & {\color[HTML]{FFC702} \textbf{/}} \\
\textbf{\begin{tabular}[c]{@{}c@{}}Web-based Python\\ Application - Server side\\ (hosted by PythonAnywhere)\end{tabular}} & {\color[HTML]{FE0000} \textbf{X}} & {\color[HTML]{32CB00} \textbf{:)}} & {\color[HTML]{32CB00} \textbf{:)}} & {\color[HTML]{32CB00} \textbf{:)}} \\
\textbf{\begin{tabular}[c]{@{}c@{}}Web-based ReactJS\\ Application - Client Side\end{tabular}} & {\color[HTML]{32CB00} \textbf{:)}} & {\color[HTML]{32CB00} \textbf{:)}} & {\color[HTML]{32CB00} \textbf{:)}} & {\color[HTML]{FFC702} \textbf{/}} \\ \hline
\end{tabular}
\caption{Consideration of experimental setups}
\label{tab:appExperiment}
\end{table}




\section{Assessment of viability in real-time video calls}
% especially focus on the time aspect for detecting the bounding box + landmark detection/shape model construction.

% Comparison dlib's shape model vs. Active Appearance Model
% \begin{quote}
%     The dlib 'Face Landmark Detection' algorithm is blazing fast, in fact it takes about 1–3ms (on desktop platform) to detect (align) a set of 68 landmarks on a given face.
% \end{quote} 
% https://medium.com/datadriveninvestor/training-alternative-dlib-shape-predictor-models-using-python-d1d8f8bd9f5c

% @InProceedings{Kazemi_2014_CVPR,
% author = {Kazemi, Vahid and Sullivan, Josephine},
% title = {One Millisecond Face Alignment with an Ensemble of Regression Trees},
% booktitle = {Proceedings of the IEEE Conference on Computer Vision and Pattern Recognition (CVPR)},
% month = {June},
% year = {2014}
% }


\chapter{Conclusion}
+ Could the research questions be answered? -> How?
+ How are the achievements of ER Applications to be interpreted?

\section{Improvements/Limitation}
AFEW-VA dataset only contains about 30.000 images. The model could be honed with more training data from different datasets and thus even perform better. This could be an improvement step for the actual implementation of such an approach. However, this step was purposefully not taken in this thesis, as it would make an objective comparison with other work very difficult.


\section{Recommendations}
\subsubsection{Next steps after this Master thesis}
=> PPI sales already contacted online banks (ING and DKB) who are very interested into ER to identify whether their costumers are interested/what intention they have.
The idea is to refine the work from this Master thesis with real-life data from the banks online consultations. This video data can be combined with a label, of whether a purchase followed the consultation. This allows to further fine-tune the proposed neural network architecture.

\subsubsection{Further areas of research}
- Making annotations more subject orientated \& culturally independent
e.g. either maybe using an EEG to detect areas of brain activity and mapping these to values of valence and arousal
e.g. actually asking subjects themselves how they felt during a video clip and using this as annotations

\subsubsection{Further fields of application}
=> Call center application for customer satisfaction
=> Enhancement of dynamic content generation through integrating a chatbot who's answers are being guided by the emotions/intentions/interest of a calling person.


\section{Outlook}
+ further fields of research
        -> combination of eye movements and physiological measurements as it was done, for example, in Noldus's product 'Cube' \citep{Noldus:2020:Facereader}
        How can this\textbf{ combination} influence the identification of human behaviors, like interest?
%%%%%%%%%%%%%%%%%%%%%%%%%%%%%%%%%%%%%%%%%%%%%%%%%%%%%%%%%%%%%

\backmatter
%Zum Einbinden des BibTeX Files:
\bibliography{literature.bib}{\nocite{*}}


\newpage
\thispagestyle{empty}
\vspace*{\fill}
\pagestyle{empty}

{\normalsize
\begin{center}\textbf{Sworn declaration}\end{center}
I hereby affirm in lieu of oath that I have independently written the present thesis in the Master's program IT Management and Consulting and that I have not used any tools other than those indicated -- especially no Internet sources not mentioned in the list of sources. All passages that have been taken literally or analogously from publications are marked as such. I further affirm that I have not previously submitted the paper in any other examination procedure and that the submitted written version corresponds to the version on the electronic storage medium.
\vspace*{1cm}\\
Hamburg, \today
\hspace*{\fill}\begin{tabular}{@{}l@{}}\hline
\makebox[5cm]{Tobias Kick}
\end{tabular}
\vspace*{3cm}

%TODO Dies ist optional, ggf. löschen!
\begin{center}\textbf{Publication}\end{center}
I agree to the placement of the work in the library of the Department of Computer Science.
\vspace*{1cm}\\
Hamburg, \today
\hspace*{\fill}\begin{tabular}{@{}l@{}}\hline
\makebox[5cm]{Tobias Kick}
\end{tabular}
}
\vspace*{\fill}

\end{document}